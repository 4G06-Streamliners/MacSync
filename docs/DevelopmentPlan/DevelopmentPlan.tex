\documentclass{article}

\usepackage{booktabs}
\usepackage{tabularx}

\title{Development Plan\\\progname}

\author{\authname}

\date{}

%% Comments

\usepackage{color}

\newif\ifcomments\commentstrue %displays comments
%\newif\ifcomments\commentsfalse %so that comments do not display

\ifcomments
\newcommand{\authornote}[3]{\textcolor{#1}{[#3 ---#2]}}
\newcommand{\todo}[1]{\textcolor{red}{[TODO: #1]}}
\else
\newcommand{\authornote}[3]{}
\newcommand{\todo}[1]{}
\fi

\newcommand{\wss}[1]{\authornote{magenta}{SS}{#1}} 
\newcommand{\plt}[1]{\authornote{cyan}{TPLT}{#1}} %For explanation of the template
\newcommand{\an}[1]{\authornote{cyan}{Author}{#1}}

%% Common Parts

\newcommand{\progname}{ProgName} % PUT YOUR PROGRAM NAME HERE
\newcommand{\authname}{Team \#, Team Name
\\ Student 1 name
\\ Student 2 name
\\ Student 3 name
\\ Student 4 name} % AUTHOR NAMES                  

\usepackage{hyperref}
    \hypersetup{colorlinks=true, linkcolor=blue, citecolor=blue, filecolor=blue,
                urlcolor=blue, unicode=false}
    \urlstyle{same}
                                


\begin{document}

\maketitle

\begin{table}[hp]
\caption{Revision History} \label{TblRevisionHistory}
\begin{tabularx}{\textwidth}{llX}
\toprule
\textbf{Date} & \textbf{Developer(s)} & \textbf{Change}\\
\midrule
Date1 & Name(s) & Description of changes\\
Date2 & Name(s) & Description of changes\\
... & ... & ...\\
\bottomrule
\end{tabularx}
\end{table}

\newpage{}

\wss{Put your introductory blurb here.  Often the blurb is a brief roadmap of
what is contained in the report.}

\wss{Additional information on the development plan can be found in the
\href{https://gitlab.cas.mcmaster.ca/courses/capstone/-/blob/main/Lectures/L02b_POCAndDevPlan/POCAndDevPlan.pdf?ref_type=heads}
{lecture slides}.}

\section{Confidential Information?}

This project does not involve the use or disclosure of confidential information 
from any industry partner. All requirements, specifications, and implementation 
details have been developed by the student team. As there is no industry-provided proprietary data 
or trade secrets involved, no Non-Disclosure Agreement (NDA) is required. 

\section{IP to Protect}

All intellectual property (IP) generated as part of this project is owned by the 
student team, in accordance with McMaster University’s policies on ownership of 
student work. Since the project is not sponsored by an external industry partner, 
no Intellectual Property Guide Acknowledgement form is required. To enable the 
intended use of the system, the student team will grant the McMaster Engineering 
Society a non-exclusive license to operate and maintain the software for its 
events and activities. Ownership of the underlying source code remains with the 
student developers. 

\section{Copyright License}

All software developed by the student team is protected under copyright law. 
To clarify permitted use, the team has adopted the MIT License, which is included 
in the project repository
\href{https://github.com/4G06-team12-MacSync/demo-repository/blob/main/LICENSE}{here}. 
This license allows reuse and modification of the code, provided that the 
original copyright notice and license terms are preserved. This ensures proper 
attribution to the development team.

\section{Team Meeting Plan}
The team will meet twice a week. Once online on Monday, from 9:30pm to 10:30pm, to discuss details on work to complete during the week. The second time will be from 5:00pm to 6:30pm to discuss progress, roadblocks, or any issues where assistance is required. This meeting will be online on the weeks there are no in-person progress check-ins and in-person otherwise. \vspace{1em}

The team will meet with the industry advisor online every week on Wednesday during the 4G06 tutorial time from 4:30pm to 5:00pm. This meeting will be to discuss project updates, clear up confusion, and ask for help if needed. \vspace{1em}

Before all meetings, an agenda will be provided. The agenda will cover the main discussion points to guide the conversation in meetings. All meetings will be structured with project updates provided by all team members first, then an update on any roadblocks or issues that need to be dealt with. The end of the meeting will focus on developing a plan for next steps, assigning tasks to each member, and defining an agenda for the next meeting. At the end of the meeting, a record of the meeting minutes will be provided by the chair of the meeting to keep the team accountable for their tasks.





\section{Team Communication Plan}

\wss{Issues on GitHub should be part of your communication plan.}

\section{Team Member Roles}

\wss{You should identify the types of roles you anticipate, like notetaker,
leader, meeting chair, reviewer.  Assigning specific people to those roles is
not necessary at this stage.  In a student team the role of the individuals will
likely change throughout the year.}

\section{Workflow Plan}

\subsection{Git Usage}
\begin{itemize}
	\item Github will be used as the primary version control system (VCS).
    \item All commits must be made to a non-master branch and submitted via a pull request before they can be merged.
    \item Branching Strategy: \begin{itemize}
    \item Main branch: Stable production-ready code waiting for a release.
    \item Documentation branches: Used to maintain and update project documentation and requirements.
        \begin{itemize}
            \item Notation: \texttt{docs/SL-<Issue\#>}
        \end{itemize}

    \item Development branch: Integration branch where all new features and bug fixes are merged. This branch will contain the latest features with the most up-to-date code. 
        \begin{itemize}
            \item Notation: \texttt{dev}
        \end{itemize}

    \item Feature branches: Used to develop and test new features before merging into the development branch. 
        \begin{itemize}
            \item Notation: \texttt{feature/SL-<Issue\#>}
        \end{itemize}

    \item Bugfix branches: Used to fix defects.
        \begin{itemize}
            \item Notation: \texttt{bugfix/SL-<Issue\#>}
        \end{itemize}
\end{itemize}
\end{itemize}

\subsection{Issues and Project Management}
\begin{itemize}
    \item Github issues will serve as the primary issue/ticket system
    \item All subsections of each document/report will have its own issue
    \item All new tasks and/or features will have its own issue
    \item Issues will be classified to using labels (e.g., \texttt{documentation}, \texttt{feature}, \texttt{bug}).
    \item Github's Kanban board will be used to monitor progress
\end{itemize}

\subsection{CI/CD Usage}
\begin{itemize}
    \item A CI/CD pipeline will be implemented using Github Actions.
    \item Continuous Integration \textbf{(CI)}:
    \begin{itemize}
        \item Automatically build the project and run tests on every pull request.
    \end{itemize}
    \item Continuous Deployment \textbf{(CD)}:
    \begin{itemize}
        \item Deploy staging builds after merging to development branch.
        \item Deploy production builds after merging to main branch.
    \end{itemize}
\end{itemize}

\section{Project Decomposition and Scheduling}

\begin{itemize}
  \item How will you be using GitHub projects?
  \item Include a link to your GitHub project
\end{itemize}

\wss{How will the project be scheduled?  This is the big picture schedule, not
details. You will need to reproduce information that is in the course outline
for deadlines.}

\section{Proof of Concept Demonstration Plan}

What is the main risk, or risks, for the success of your project?  What will you
demonstrate during your proof of concept demonstration to convince yourself that
you will be able to overcome this risk?

\section{Expected Technology}

\wss{What programming language or languages do you expect to use?  What external
libraries?  What frameworks?  What technologies.  Are there major components of
the implementation that you expect you will implement, despite the existence of
libraries that provide the required functionality.  For projects with machine
learning, will you use pre-trained models, or be training your own model?  }

\wss{The implementation decisions can, and likely will, change over the course
of the project.  The initial documentation should be written in an abstract way;
it should be agnostic of the implementation choices, unless the implementation
choices are project constraints.  However, recording our initial thoughts on
implementation helps understand the challenge level and feasibility of a
project.  It may also help with early identification of areas where project
members will need to augment their training.}

Topics to discuss include the following:

\begin{itemize}
\item Specific programming language
\item Specific libraries
\item Pre-trained models
\item Specific linter tool (if appropriate)
\item Specific unit testing framework
\item Investigation of code coverage measuring tools
\item Specific plans for Continuous Integration (CI), or an explanation that CI
  is not being done
\item Specific performance measuring tools (like Valgrind), if
  appropriate
\item Tools you will likely be using?
\end{itemize}

\wss{git, GitHub and GitHub projects should be part of your technology.}

\section{Coding Standard}

The team will adopt consistent coding standards to ensure readability, maintainability, and effective collaboration. For JavaScript code, the team will follow the \href{https://github.com/airbnb/javascript}{Airbnb JavaScript Style Guide}, which provides well-established conventions for formatting and best practices. Code formatting will be automated with \textbf{Prettier}, and linting will be enforced with \textbf{ESLint} to reduce inconsistencies across contributions.

Naming conventions will remain uniform across the stack:
\begin{itemize}
  \item \texttt{camelCase} for variables and functions
  \item \texttt{PascalCase} for React components and NestJS classes
  \item \texttt{snake\_case} for PostgreSQL tables and column names
\end{itemize}

Beyond style rules, the team will apply \textbf{Clean Code principles} to keep the codebase simple and easy to understand. This means writing small and focused functions, choosing descriptive names, avoiding duplication, and keeping modules cohesive. Comments will be added only where necessary to clarify complex logic, with the preference being self-explanatory code.

For version control, all work will be done in feature branches, with pull requests reviewed before merging into \texttt{staging} or \texttt{main}. Commit messages will follow the imperative style (for example, “Add payment processing logic”) to maintain a clear project history.

\newpage{}

\section*{Appendix --- Reflection}

\wss{Not required for CAS 741}

The purpose of reflection questions is to give you a chance to assess your own
learning and that of your group as a whole, and to find ways to improve in the
future. Reflection is an important part of the learning process.  Reflection is
also an essential component of a successful software development process.  

Reflections are most interesting and useful when they're honest, even if the
stories they tell are imperfect. You will be marked based on your depth of
thought and analysis, and not based on the content of the reflections
themselves. Thus, for full marks we encourage you to answer openly and honestly
and to avoid simply writing ``what you think the evaluator wants to hear.''

Please answer the following questions.  Some questions can be answered on the
team level, but where appropriate, each team member should write their own
response:


\begin{enumerate}
    \item Why is it important to create a development plan prior to starting the
    project?
    \item In your opinion, what are the advantages and disadvantages of using
    CI/CD?
    \item What disagreements did your group have in this deliverable, if any,
    and how did you resolve them?
\end{enumerate}

\newpage{}

\section*{Appendix --- Team Charter}

\subsection*{External Goals}
\begin{itemize}
    \item Secure a grade of A+ in 4G06.
    \item Gain real-world experience in areas of project management, teamwork and shipping projects.
    \item Create a strong portfolio project that has a large user base and can be discussed in interviews.  
\end{itemize}


\subsection*{Attendance}

\subsubsection*{Expectations}
\begin{itemize}
    \item All members are expected to attend all scheduled meetings either in-person or online.
    \item All members are expected to be on time for meetings and remain present for the duration of the meeting.
    \item All members are expected to notify the team 24 hours beforehand if they are unable to attend meetings.
\end{itemize}

\subsubsection*{Acceptable Excuse}
Acceptable excuses for missing meetings or deadlines include the following:
\begin{itemize}
    \item Family emergency
    \item Health issues
    \item Medical appointments
\end{itemize}
Unacceptable excuses for missing meetings or deadlines include the following:
\begin{itemize}
    \item Forgetting about meetings
    \item Oversleeping
\end{itemize}

\subsubsection*{In Case of Emergency}
\begin{itemize}
    \item If the emergency affects the members ability to attend a meeting, they must notify the group immediately or as soon as possible after the emergency.
    \item If the emergency affects completion of work, the member must communicate explicitly what work is done and what remains so the team can decide on the division of labour or a plan to catch-up.
\end{itemize}


\subsection*{Accountability and Teamwork}

\subsubsection*{Quality} 
\begin{itemize}
    \item All members must come to meetings prepared with progress updates on their tasks.
    \item Deliverables should be well-organized and reviewed before they are shared with the team.
    \item Contributions in the code repository must follow team standards for readability and documentation.
\end{itemize}

\subsubsection*{Attitude}
\begin{itemize}
    \item All members must listen actively to all other members and avoid passive or dismissive behavior.
    \item All members should respect all ideas and viewpoints without judgment.
    \item All members are encouraged to share their ideas openly.
    \item All members must adopt a simple code of conduct revolving on principles of respect, communication and collaboration.
    \item Conflicts must be resolved respectfully and if unresolved will be escalated to the TA.
\end{itemize}

\subsubsection*{Stay on Track}
\begin{itemize}
    \item Weekly progress check-ins where members discuss what they completed and next steps will be used to keep the team on track. 
    \item GitHub Issues and commits will be used to ensure members contribute as expected.
    
    \item Metrics: \begin{itemize}
        \item An attendance rate of 100\% is expected from all members unless an excuse is deemed valid and acceptable by the team.
        \item Consistent GitHub activity is expected per week with issues in progress or items being committed for review.
        \item On-time delivery of tasks on assigned deadlines.
    \end{itemize}
    \item Rewards: \begin{itemize}
        \item Members who reach targets early or who do well overall will be rewarded with acknowledgments of being strong contributors. \end{itemize}
    \item Consequences: \begin{itemize}
        \item First time incident will be a verbal reminder.
        \item Second time will result in a discussion with the TA.
        \item Third time and more will result in escalation to the course instructor.
    \end{itemize}
\end{itemize}

\subsubsection*{Team Building}
\begin{itemize}
    \item Start meetings with small talk and conversations to get to know each other and maintain team connection.
    \item Doing a group ritual of small activities like board games or lunch to celebrate milestone deliveries.
\end{itemize}

\subsubsection*{Decision Making} 
\begin{itemize}
    \item Initial plan will be to reach consensus when possible.
    \item If consensus cannot be reached, opt for a majority vote.
    \item If disagreements still exist within the team if consensus and majority vote leaves members unsatisfied, involve the TA for handling disagreements.
\end{itemize}


\end{document}