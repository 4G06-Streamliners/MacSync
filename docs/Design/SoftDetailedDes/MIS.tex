\documentclass[12pt, titlepage]{article}

\usepackage{amsmath, mathtools}

\usepackage[round]{natbib}
\usepackage{amsfonts}
\usepackage{amssymb}
\usepackage{graphicx}
\usepackage{colortbl}
\usepackage{xr}
\usepackage{hyperref}
\usepackage{longtable}
\usepackage{xfrac}
\usepackage{tabularx}
\usepackage{float}
\usepackage{siunitx}
\usepackage{booktabs}
\usepackage{multirow}
\usepackage[section]{placeins}
\usepackage{caption}
\usepackage{fullpage}

\hypersetup{
bookmarks=true,     % show bookmarks bar?
colorlinks=true,       % false: boxed links; true: colored links
linkcolor=red,          % color of internal links (change box color with linkbordercolor)
citecolor=blue,      % color of links to bibliography
filecolor=magenta,  % color of file links
urlcolor=cyan          % color of external links
}

\usepackage{array}

\externaldocument{../../SRS/SRS}

\input{../../Comments}
%% Common Parts

\newcommand{\progname}{Software Engineering} % PUT YOUR PROGRAM NAME HERE
\newcommand{\authname}{Team \#12, Streamliners
\\ Mahad Ahmed
\\ Abyan Jaigirdar
\\ Prerna Prabhu
\\ Farhan Rahman
\\ Ali Zia} % AUTHOR NAMES                  

\usepackage{hyperref}
    \hypersetup{colorlinks=true, linkcolor=blue, citecolor=blue, filecolor=blue,
                urlcolor=blue, unicode=false}
    \urlstyle{same}     

\begin{document}

\title{Module Interface Specification for \progname{}}

\author{\authname}

\date{\today}

\maketitle

\pagenumbering{roman}

\section{Revision History}

\begin{tabularx}{\textwidth}{p{3cm}p{2cm}X}
\toprule {\bf Date} & {\bf Version} & {\bf Notes}\\
\midrule
Date 1 & 1.0 & Notes\\
Date 2 & 1.1 & Notes\\
\bottomrule
\end{tabularx}

~\newpage

\section{Symbols, Abbreviations and Acronyms}

See SRS Documentation at \wss{give url}

\wss{Also add any additional symbols, abbreviations or acronyms}

\newpage

\tableofcontents

\newpage

\pagenumbering{arabic}

\section{Introduction}

The following document details the Module Interface Specifications for
\wss{Fill in your project name and description}

Complementary documents include the System Requirement Specifications
and Module Guide.  The full documentation and implementation can be
found at \url{...}.  \wss{provide the url for your repo}

\section{Notation}

\wss{You should describe your notation.  You can use what is below as
  a starting point.}

The structure of the MIS for modules comes from \citet{HoffmanAndStrooper1995},
with the addition that template modules have been adapted from
\cite{GhezziEtAl2003}.  The mathematical notation comes from Chapter 3 of
\citet{HoffmanAndStrooper1995}.  For instance, the symbol := is used for a
multiple assignment statement and conditional rules follow the form $(c_1
\Rightarrow r_1 | c_2 \Rightarrow r_2 | ... | c_n \Rightarrow r_n )$.

The following table summarizes the primitive data types used by \progname. 

\begin{center}
\renewcommand{\arraystretch}{1.2}
\noindent 
\begin{tabular}{l l p{7.5cm}} 
\toprule 
\textbf{Data Type} & \textbf{Notation} & \textbf{Description}\\ 
\midrule
character & char & a single symbol or digit\\
integer & $\mathbb{Z}$ & a number without a fractional component in (-$\infty$, $\infty$) \\
natural number & $\mathbb{N}$ & a number without a fractional component in [1, $\infty$) \\
real & $\mathbb{R}$ & any number in (-$\infty$, $\infty$)\\
\bottomrule
\end{tabular} 
\end{center}

\noindent
The specification of \progname \ uses some derived data types: sequences, strings, and
tuples. Sequences are lists filled with elements of the same data type. Strings
are sequences of characters. Tuples contain a list of values, potentially of
different types. In addition, \progname \ uses functions, which
are defined by the data types of their inputs and outputs. Local functions are
described by giving their type signature followed by their specification.

\section{Module Decomposition}

The following table is taken directly from the Module Guide document for this project.

\begin{table}[h!]
\centering
\begin{tabular}{p{0.3\textwidth} p{0.6\textwidth}}
\toprule
\textbf{Level 1} & \textbf{Level 2}\\
\midrule

{Hardware-Hiding} & ~ \\
\midrule

\multirow{7}{0.3\textwidth}{Behaviour-Hiding} & Input Parameters\\
& Output Format\\
& Output Verification\\
& Temperature ODEs\\
& Energy Equations\\ 
& Control Module\\
& Specification Parameters Module\\
\midrule

\multirow{3}{0.3\textwidth}{Software Decision} & {Sequence Data Structure}\\
& ODE Solver\\
& Plotting\\
\bottomrule

\end{tabular}
\caption{Module Hierarchy}
\label{TblMH}
\end{table}

\newpage
~\newpage


\newpage
\section{Module Interface Specifications} \label{SecMIS}

This section provides the Module Interface Specifications (MIS) for each module 
defined in the Module Guide. Each module is documented independently, following 
the notation and structure described in Section 2.

\subsection{MIS of Payment Processing Module (M1)} \label{MIS_M1}

\subsubsection{Module}

Payment Processing Module (M1)

This module prepares and submits payment requests from the client side.  
It validates the payment data, attaches authentication headers, and communicates 
with the backend API that integrates with Stripe. The module does not handle 
any payment verification or charge confirmation locally. Instead, verification 
is performed by the backend through Stripe’s secure payment APIs, configured 
in the Payment Configuration Module (M4).

\subsubsection{Uses}

\begin{itemize}
    \item API Layer (HTTP communication with backend)
    \item User Authorization Module (Frontend) (M8F)
    \item Payment Configuration Module (M4) (indirect, via backend)
\end{itemize}

\subsection{Syntax}

\subsubsection{Exported Constants}

None.

\subsubsection{Exported Access Programs}

\begin{center}
\begin{tabular}{p{3cm} p{4cm} p{4cm} p{2cm}}
\hline
\textbf{Name} & \textbf{In} & \textbf{Out} & \textbf{Exceptions} \\
\hline
submitPayment & PaymentInfo & ConfirmationResponse & InvalidPaymentData, NetworkError \\
\hline
validatePaymentData & PaymentInfo & Boolean & InvalidPaymentData \\
\hline
\end{tabular}
\end{center}

\subsection{Semantics}

\subsubsection{State Variables}

None.

\subsubsection{Environment Variables}

\begin{itemize}
    \item Network connection to the backend API
    \item AuthToken from M8F
\end{itemize}

\subsubsection{Assumptions}

\begin{itemize}
    \item User must be authenticated before submitting a payment.
    \item Device must have a stable network connection.
    \item The backend exposes a REST endpoint \texttt{POST /api/payments/create-intent}
          for initializing Stripe PaymentIntents.
    \item The backend handles all Stripe secret keys and secure payment operations.
\end{itemize}

\subsubsection{Access Routine Semantics}

\noindent \textbf{submitPayment}(paymentInfo):

\begin{itemize}
    \item transition: None.
    \item output: A ConfirmationResponse containing the client secret, payment result, 
    or any backend-supplied status (e.g., requires\_action, succeeded, failed).
    \item exception: InvalidPaymentData if required fields are missing or malformed; 
    NetworkError if the request fails.
\end{itemize}

\vspace{0.2cm}

\noindent \textbf{validatePaymentData}(paymentInfo):

\begin{itemize}
    \item transition: None.
    \item output: Returns true if all required fields (amount, event ID, billing details) 
    are correctly defined.
    \item exception: InvalidPaymentData if fields are empty or violate basic constraints.
\end{itemize}

\subsubsection{Local Functions}

\begin{itemize}
    \item formatRequestBody(paymentInfo): builds the JSON payload for creating a Stripe PaymentIntent.
    \item attachAuthHeader(): retrieves user Authorization token from M8F and attaches it 
          to the API request.
    \item handleApiResponse(response): converts backend JSON into a ConfirmationResponse 
          (e.g., extracting client\_secret or error information).
\end{itemize}

\newpage

\bibliographystyle {plainnat}
\bibliography {../../../refs/References}

\newpage

\section{Appendix} \label{Appendix}

\wss{Extra information if required}

\newpage{}

\section*{Appendix --- Reflection}

\wss{Not required for CAS 741 projects}

The information in this section will be used to evaluate the team members on the
graduate attribute of Problem Analysis and Design.

\input{../../Reflection.tex}

\begin{enumerate}
  \item What went well while writing this deliverable? 
  \item What pain points did you experience during this deliverable, and how
    did you resolve them?
  \item Which of your design decisions stemmed from speaking to your client(s)
  or a proxy (e.g. your peers, stakeholders, potential users)? For those that
  were not, why, and where did they come from?
  \item While creating the design doc, what parts of your other documents (e.g.
  requirements, hazard analysis, etc), it any, needed to be changed, and why?
  \item What are the limitations of your solution?  Put another way, given
  unlimited resources, what could you do to make the project better? (LO\_ProbSolutions)
  \item Give a brief overview of other design solutions you considered.  What
  are the benefits and tradeoffs of those other designs compared with the chosen
  design?  From all the potential options, why did you select the documented design?
  (LO\_Explores)
\end{enumerate}


\end{document}