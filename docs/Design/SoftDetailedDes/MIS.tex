\documentclass[12pt, titlepage]{article}

\usepackage{amsmath, mathtools}

\usepackage[round]{natbib}
\usepackage{amsfonts}
\usepackage{amssymb}
\usepackage{graphicx}
\usepackage{colortbl}
\usepackage{xr}
\usepackage{hyperref}
\usepackage{longtable}
\usepackage{xfrac}
\usepackage{tabularx}
\usepackage{float}
\usepackage{siunitx}
\usepackage{booktabs}
\usepackage{multirow}
\usepackage[section]{placeins}
\usepackage{caption}
\usepackage{fullpage}

\hypersetup{
bookmarks=true,     % show bookmarks bar?
colorlinks=true,       % false: boxed links; true: colored links
linkcolor=red,          % color of internal links (change box color with linkbordercolor)
citecolor=blue,      % color of links to bibliography
filecolor=magenta,  % color of file links
urlcolor=cyan          % color of external links
}

\usepackage{array}

\externaldocument{../../SRS/SRS}

\input{../../Comments}
%% Common Parts

\newcommand{\progname}{Software Engineering} % PUT YOUR PROGRAM NAME HERE
\newcommand{\authname}{Team \#12, Streamliners
\\ Mahad Ahmed
\\ Abyan Jaigirdar
\\ Prerna Prabhu
\\ Farhan Rahman
\\ Ali Zia} % AUTHOR NAMES                  

\usepackage{hyperref}
    \hypersetup{colorlinks=true, linkcolor=blue, citecolor=blue, filecolor=blue,
                urlcolor=blue, unicode=false}
    \urlstyle{same}     

\begin{document}

\title{Module Interface Specification for \progname{}}

\author{\authname}

\date{\today}

\maketitle

\pagenumbering{roman}

\section{Revision History}

\begin{tabularx}{\textwidth}{p{3cm}p{2cm}X}
\toprule {\bf Date} & {\bf Version} & {\bf Notes}\\
\midrule
Date 1 & 1.0 & Notes\\
Date 2 & 1.1 & Notes\\
\bottomrule
\end{tabularx}

~\newpage

\section{Symbols, Abbreviations and Acronyms}

See SRS Documentation at \wss{give url}

\wss{Also add any additional symbols, abbreviations or acronyms}

\newpage

\tableofcontents

\newpage

\pagenumbering{arabic}

\section{Introduction}

The following document details the Module Interface Specifications for
\wss{Fill in your project name and description}

Complementary documents include the System Requirement Specifications
and Module Guide.  The full documentation and implementation can be
found at \url{...}.  \wss{provide the url for your repo}

\section{Notation}

\wss{You should describe your notation.  You can use what is below as
  a starting point.}

The structure of the MIS for modules comes from \citet{HoffmanAndStrooper1995},
with the addition that template modules have been adapted from
\cite{GhezziEtAl2003}.  The mathematical notation comes from Chapter 3 of
\citet{HoffmanAndStrooper1995}.  For instance, the symbol := is used for a
multiple assignment statement and conditional rules follow the form $(c_1
\Rightarrow r_1 | c_2 \Rightarrow r_2 | ... | c_n \Rightarrow r_n )$.

The following table summarizes the primitive data types used by \progname. 

\begin{center}
\renewcommand{\arraystretch}{1.2}
\noindent 
\begin{tabular}{l l p{7.5cm}} 
\toprule 
\textbf{Data Type} & \textbf{Notation} & \textbf{Description}\\ 
\midrule
character & char & a single symbol or digit\\
integer & $\mathbb{Z}$ & a number without a fractional component in (-$\infty$, $\infty$) \\
natural number & $\mathbb{N}$ & a number without a fractional component in [1, $\infty$) \\
real & $\mathbb{R}$ & any number in (-$\infty$, $\infty$)\\
\bottomrule
\end{tabular} 
\end{center}

\noindent
The specification of \progname \ uses some derived data types: sequences, strings, and
tuples. Sequences are lists filled with elements of the same data type. Strings
are sequences of characters. Tuples contain a list of values, potentially of
different types. In addition, \progname \ uses functions, which
are defined by the data types of their inputs and outputs. Local functions are
described by giving their type signature followed by their specification.

\section{Module Decomposition}

The following table is taken directly from the Module Guide document for this project.

\begin{table}[h!]
\centering
\begin{tabular}{p{0.3\textwidth} p{0.6\textwidth}}
\toprule
\textbf{Level 1} & \textbf{Level 2}\\
\midrule

{Hardware-Hiding} & ~ \\
\midrule

\multirow{7}{0.3\textwidth}{Behaviour-Hiding} & Input Parameters\\
& Output Format\\
& Output Verification\\
& Temperature ODEs\\
& Energy Equations\\ 
& Control Module\\
& Specification Parameters Module\\
\midrule

\multirow{3}{0.3\textwidth}{Software Decision} & {Sequence Data Structure}\\
& ODE Solver\\
& Plotting\\
\bottomrule

\end{tabular}
\caption{Module Hierarchy}
\label{TblMH}
\end{table}

\newpage
~\newpage

\section{MIS of Payment Configuration Module (M4)}

\subsection{Module}

\texttt{PaymentConfigurationModule}\\

\noindent
This module centralizes all configuration required for secure payment operations.
It exposes safe, restricted payment settings to other modules such as the Payment
Processing Module (M1) while hiding sensitive Stripe configuration logic and
backend API wiring. It ensures a single source of truth for currency rules,
tax configuration, receipt formatting, and API endpoint definitions related to
payments.

\subsection{Uses}

\begin{itemize}
    \item Payment Processing Module (M1): Uses exported configuration values
    to construct valid payment requests and interact with backend payment endpoints.
    \item User Authorization Module (M8): Provides authentication tokens required
    when modules call backend payment APIs.
    \item Access Control Module (M5): Ensures that only authorized user roles may
    access payment configuration or initiate restricted payment operations.
    \item Notification Handling Module (M7): Uses configured receipt and invoice
    formatting rules when sending payment confirmations.
    \item Backend Stripe Integration Service: Provides the actual
    PaymentIntent creation, confirmation, and refund endpoints (e.g., 
    \texttt{/api/payments}, \texttt{/api/payments/refund}).
\end{itemize}


\subsection{Syntax}

\subsubsection{Exported Constants}

\begin{itemize}
    \item \texttt{STRIPE\_PUBLIC\_KEY : Str}\\
    The publishable Stripe key used by the frontend; never contains secret keys.
    \item \texttt{PAYMENT\_API\_BASE : URL}\\
    Base backend endpoint for payment operations.
    \item \texttt{CURRENCY\_CODE : Str}\\
    Currency used for all events (\texttt{CAD}).
    \item \texttt{TAX\_RATE : Float}\\
    The global tax percentage applied to event payments.
    \item \texttt{INTENT\_CACHE\_TTL\_SEC : Nat}\\
    TTL for caching Stripe PaymentIntent client secrets.
\end{itemize}

\subsubsection{Exported Access Programs}

\begin{center}
\renewcommand{\arraystretch}{1.2}
\begin{tabularx}{\textwidth}{l l l X}
\toprule
\textbf{Name} & \textbf{In} & \textbf{Out} & \textbf{Exceptions} \\
\midrule
\texttt{getPublicStripeKey} & -- & Str & KeyNotAvailable \\
\texttt{getPaymentAPIEndpoint} & OperationType & URL & InvalidOperationType \\
\texttt{getCurrencyConfig} & -- & CurrencyConfig & -- \\
\texttt{shouldRefreshPaymentIntent} & LastUpdated: Time & Bool & -- \\
\texttt{getReceiptConfig} & -- & ReceiptConfig & -- \\
\bottomrule
\end{tabularx}
\end{center}

\subsection{Semantics}

\subsubsection{State Variables}

\begin{itemize}
    \item \texttt{publicKey : Str}
    \item \texttt{paymentAPIMap : dict[OperationType $\rightarrow$ URL]}
    \item \texttt{currencyConfig : CurrencyConfig}
    \item \texttt{receiptConfig : ReceiptConfig}
    \item \texttt{cacheTTL : Nat}
\end{itemize}

\subsubsection{Environment Variables}

\begin{itemize}
    \item \texttt{ENV\_STRIPE\_PUBLIC\_KEY : Str} - Loaded from build-time environment.
    \item \texttt{ENV\_PAYMENT\_API\_BASE : URL} - Backend service base URL.
    \item \texttt{SystemTime : Time} - Used for TTL and expiration checks.
\end{itemize}

\subsubsection{Assumptions}

\begin{itemize}
    \item Stripe secret keys are stored only on the backend and never exposed to the client.
    \item Backend exposes the routes \texttt{/api/payments}, 
    \texttt{/api/payments/confirm}, and \texttt{/api/payments/refund}.
    \item All callers of this module have been authenticated by M8 and authorized by M5.
    \item Configuration is loaded once per application session.
\end{itemize}

\subsubsection{Access Routine Semantics}

\noindent\textbf{\texttt{getPublicStripeKey()}}
\begin{itemize}
\item transition: none
\item output: \texttt{publicKey}
\item exception: \texttt{KeyNotAvailable} if key is missing or invalid.
\end{itemize}

\vspace{0.2cm}

\noindent\textbf{\texttt{getPaymentAPIEndpoint(opType)}}
\begin{itemize}
\item transition: none
\item output: returns URL mapped to the given operation type.
\item exception: \texttt{InvalidOperationType} if no mapping is defined.
\end{itemize}

\vspace{0.2cm}

\noindent\textbf{\texttt{getCurrencyConfig()}}
\begin{itemize}
\item transition: none
\item output: returns currency and tax configuration.
\item exception: none.
\end{itemize}

\vspace{0.2cm}

\noindent\textbf{\texttt{shouldRefreshPaymentIntent(lastUpdated)}}
\begin{itemize}
\item transition: none
\item output: \texttt{true} if the PaymentIntent cache TTL has expired.
\item exception: none.
\end{itemize}

\vspace{0.2cm}

\noindent\textbf{\texttt{getReceiptConfig()}}
\begin{itemize}
\item transition: none
\item output: returns canonical receipt formatting settings for notifications.
\item exception: none.
\end{itemize}

\subsubsection{Local Functions}

\begin{itemize}
    \item \texttt{loadEnvValue(key)}: safely retrieves build-time environment variables.
    \item \texttt{buildEndpoint(base, suffix)}: constructs consistent backend payment URLs.
    \item \texttt{validatePublicKey(str)}: checks prefix/type of Stripe public key.
    \item \texttt{computeTTL(t)}: helper for determining TTL expiration.
\end{itemize}


\newpage

\bibliographystyle {plainnat}
\bibliography {../../../refs/References}

\newpage

\section{Appendix} \label{Appendix}

\wss{Extra information if required}

\newpage{}

\section*{Appendix --- Reflection}

\wss{Not required for CAS 741 projects}

The information in this section will be used to evaluate the team members on the
graduate attribute of Problem Analysis and Design.

\input{../../Reflection.tex}

\begin{enumerate}
  \item What went well while writing this deliverable? 
  \item What pain points did you experience during this deliverable, and how
    did you resolve them?
  \item Which of your design decisions stemmed from speaking to your client(s)
  or a proxy (e.g. your peers, stakeholders, potential users)? For those that
  were not, why, and where did they come from?
  \item While creating the design doc, what parts of your other documents (e.g.
  requirements, hazard analysis, etc), it any, needed to be changed, and why?
  \item What are the limitations of your solution?  Put another way, given
  unlimited resources, what could you do to make the project better? (LO\_ProbSolutions)
  \item Give a brief overview of other design solutions you considered.  What
  are the benefits and tradeoffs of those other designs compared with the chosen
  design?  From all the potential options, why did you select the documented design?
  (LO\_Explores)
\end{enumerate}


\end{document}