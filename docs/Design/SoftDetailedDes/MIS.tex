\documentclass[12pt, titlepage]{article}

\usepackage{amsmath, mathtools}

\usepackage[round]{natbib}
\usepackage{amsfonts}
\usepackage{amssymb}
\usepackage{graphicx}
\usepackage{colortbl}
\usepackage{xr}
\usepackage{hyperref}
\usepackage{longtable}
\usepackage{xfrac}
\usepackage{tabularx}
\usepackage{float}
\usepackage{siunitx}
\usepackage{booktabs}
\usepackage{multirow}
\usepackage[section]{placeins}
\usepackage{caption}
\usepackage{fullpage}

\hypersetup{
bookmarks=true,     % show bookmarks bar?
colorlinks=true,       % false: boxed links; true: colored links
linkcolor=red,          % color of internal links (change box color with linkbordercolor)
citecolor=blue,      % color of links to bibliography
filecolor=magenta,  % color of file links
urlcolor=cyan          % color of external links
}

\usepackage{array}

\externaldocument{../../SRS/SRS}

\input{../../Comments}
%% Common Parts

\newcommand{\progname}{Software Engineering} % PUT YOUR PROGRAM NAME HERE
\newcommand{\authname}{Team \#12, Streamliners
\\ Mahad Ahmed
\\ Abyan Jaigirdar
\\ Prerna Prabhu
\\ Farhan Rahman
\\ Ali Zia} % AUTHOR NAMES                  

\usepackage{hyperref}
    \hypersetup{colorlinks=true, linkcolor=blue, citecolor=blue, filecolor=blue,
                urlcolor=blue, unicode=false}
    \urlstyle{same}     

\begin{document}

\title{Module Interface Specification for \progname{}}

\author{\authname}

\date{\today}

\maketitle

\pagenumbering{roman}

\section{Revision History}

\begin{tabularx}{\textwidth}{p{3cm}p{2cm}X}
\toprule {\bf Date} & {\bf Version} & {\bf Notes}\\
\midrule
Date 1 & 1.0 & Notes\\
Date 2 & 1.1 & Notes\\
\bottomrule
\end{tabularx}

~\newpage

\section{Symbols, Abbreviations and Acronyms}

See SRS Documentation at \wss{give url}

\wss{Also add any additional symbols, abbreviations or acronyms}

\newpage

\tableofcontents

\newpage

\pagenumbering{arabic}

\section{Introduction}

The following document details the Module Interface Specifications for
\wss{Fill in your project name and description}

Complementary documents include the System Requirement Specifications
and Module Guide.  The full documentation and implementation can be
found at \url{...}.  \wss{provide the url for your repo}

\section{Notation}

\wss{You should describe your notation.  You can use what is below as
  a starting point.}

The structure of the MIS for modules comes from \citet{HoffmanAndStrooper1995},
with the addition that template modules have been adapted from
\cite{GhezziEtAl2003}.  The mathematical notation comes from Chapter 3 of
\citet{HoffmanAndStrooper1995}.  For instance, the symbol := is used for a
multiple assignment statement and conditional rules follow the form $(c_1
\Rightarrow r_1 | c_2 \Rightarrow r_2 | ... | c_n \Rightarrow r_n )$.

The following table summarizes the primitive data types used by \progname. 

\begin{center}
\renewcommand{\arraystretch}{1.2}
\noindent 
\begin{tabular}{l l p{7.5cm}} 
\toprule 
\textbf{Data Type} & \textbf{Notation} & \textbf{Description}\\ 
\midrule
character & char & a single symbol or digit\\
integer & $\mathbb{Z}$ & a number without a fractional component in (-$\infty$, $\infty$) \\
natural number & $\mathbb{N}$ & a number without a fractional component in [1, $\infty$) \\
real & $\mathbb{R}$ & any number in (-$\infty$, $\infty$)\\
\bottomrule
\end{tabular} 
\end{center}

\noindent
The specification of \progname \ uses some derived data types: sequences, strings, and
tuples. Sequences are lists filled with elements of the same data type. Strings
are sequences of characters. Tuples contain a list of values, potentially of
different types. In addition, \progname \ uses functions, which
are defined by the data types of their inputs and outputs. Local functions are
described by giving their type signature followed by their specification.

\section{Module Decomposition}

The following table is taken directly from the Module Guide document for this project.

\begin{table}[h!]
\centering
\begin{tabular}{p{0.3\textwidth} p{0.6\textwidth}}
\toprule
\textbf{Level 1} & \textbf{Level 2}\\
\midrule

{Hardware-Hiding} & ~ \\
\midrule

\multirow{7}{0.3\textwidth}{Behaviour-Hiding} & Input Parameters\\
& Output Format\\
& Output Verification\\
& Temperature ODEs\\
& Energy Equations\\ 
& Control Module\\
& Specification Parameters Module\\
\midrule

\multirow{3}{0.3\textwidth}{Software Decision} & {Sequence Data Structure}\\
& ODE Solver\\
& Plotting\\
\bottomrule

\end{tabular}
\caption{Module Hierarchy}
\label{TblMH}
\end{table}

\newpage
~\newpage

\section{MIS of User Authorization Module (M8)}

\subsection{Module}

\texttt{UserAuthorizationModule}\\

\noindent
This module provides authentication and identity verification for all other 
modules in the system. It manages login sessions, token issuance, token 
validation, and user identity retrieval. The module integrates with external 
authentication providers when applicable and acts as the system’s source of 
truth for authenticated user identity data used by RBAC/FBAC (M5/M2) and any 
module requiring secure access.

\subsection{Uses}

\begin{itemize}
    \item Access Control Module (M5): Uses identity and token validation functions to determine a user's permissions.
    \item Payment Processing Module (M1): Requires a verified user identity before initiating payment or refund operations.
    \item Sign-Up Module (M3): Uses authenticated user identifiers when  creating bus/table/RSVP sign-up records.
    \item Notification Handling Module (M7): Uses authenticated user information to route notifications to the correct recipient.
    \item External Authentication Provider (e.g., OAuth/SSO): Provides the 
    underlying authentication mechanism for secure login.
\end{itemize}

\subsection{Syntax}

\subsubsection{Exported Constants}

\begin{itemize}
    \item \texttt{TOKEN\_EXPIRY\_MIN : Nat}\\
    The number of minutes before an issued access token expires.
    \item \texttt{REFRESH\_TOKEN\_SUPPORTED : Bool}\\
    Indicates whether refresh tokens are enabled in the current 
    deployment configuration.
\end{itemize}

\subsubsection{Exported Access Programs}

\begin{center}
\renewcommand{\arraystretch}{1.2}
\begin{tabularx}{\textwidth}{l l l X}
\toprule
\textbf{Name} & \textbf{In} & \textbf{Out} & \textbf{Exceptions} \\
\midrule
\texttt{login} & Credentials & AuthToken & InvalidCredentials, NetworkError \\
\texttt{logout} & AuthToken & Bool & TokenError \\
\texttt{validateToken} & AuthToken & Bool & TokenExpired, TokenInvalid \\
\texttt{getUserInfo} & AuthToken & UserProfile & TokenInvalid, NetworkError \\
\texttt{refreshToken} & RefreshToken & AuthToken & TokenInvalid, NotSupported \\
\bottomrule
\end{tabularx}
\end{center}

\subsection{Semantics}

\subsubsection{State Variables}

\begin{itemize}
    \item \texttt{activeSessions : dict[token $\rightarrow$ UserID]} \\
    Stores active session mappings for all authenticated users.
    \item \texttt{tokenExpiryTimes : dict[token $\rightarrow$ Time]} \\
    Tracks expiration timestamps for each active token.
\end{itemize}

\subsubsection{Environment Variables}

\begin{itemize}
    \item \texttt{AUTH\_SERVER\_URL : URL} \\
    External authentication provider endpoint (OAuth/SSO).
    \item \texttt{SystemTime : Time} \\
    Used to validate token expiration.
\end{itemize}

\subsubsection{Assumptions}

\begin{itemize}
    \item All modules calling this one must supply a token obtained from \texttt{login} or \texttt{refreshToken}.
    \item The external authentication provider (OAuth/SSO) is available and reliable.
    \item Tokens are cryptographically signed by the authentication provider and cannot be tampered with by the client.
    \item Session expiry rules follow the SRS and Hazard Analysis requirements (e.g., 15-minute admin timeout).
\end{itemize}

\subsubsection{Access Routine Semantics}

\noindent\textbf{\texttt{login(credentials)}}
\begin{itemize}
\item transition: Adds new entry to \texttt{activeSessions} and \texttt{tokenExpiryTimes}.
\item output: Returns a newly issued \texttt{AuthToken}.
\item exception: \texttt{InvalidCredentials} if credentials rejected or \texttt{NetworkError} if authentication provider is unreachable.
\end{itemize}

\vspace{0.2cm}

\noindent\textbf{\texttt{logout(token)}}
\begin{itemize}
\item transition: Removes \texttt{token} from all session maps.
\item output: \texttt{true} on success.
\item exception: \texttt{TokenError} if token is not active.
\end{itemize}

\vspace{0.2cm}

\noindent\textbf{\texttt{validateToken(token)}}
\begin{itemize}
\item transition: none.
\item output: \texttt{true} if token is valid and unexpired.
\item exception: \texttt{TokenInvalid} or \texttt{TokenExpired}.
\end{itemize}

\vspace{0.2cm}

\noindent\textbf{\texttt{getUserInfo(token)}}
\begin{itemize}
\item transition: none.
\item output: Returns the authenticated user's profile (ID, email, name, role).
\item exception: \texttt{TokenInvalid} or \texttt{NetworkError}.
\end{itemize}

\vspace{0.2cm}

\noindent\textbf{\texttt{refreshToken(refreshToken)}}
\begin{itemize}
\item transition: Replaces expired/expiring token with new token, if supported.
\item output: Returns newly issued \texttt{AuthToken}.
\item exception: \texttt{TokenInvalid} if the supplied refresh token is invalid, \texttt{NotSupported} if refresh tokens are disabled.
\end{itemize}

\subsubsection{Local Functions}

\begin{itemize}
    \item \texttt{decodeToken(token)}: Verifies token signature and extracts identity data.
    \item \texttt{isExpired(token)}: Checks whether current time exceeds stored expiry.
    \item \texttt{requestAuthServer(endpoint, payload)}: Handles secure communication with the external authentication provider.
\end{itemize}


\newpage

\bibliographystyle {plainnat}
\bibliography {../../../refs/References}

\newpage

\section{Appendix} \label{Appendix}

\wss{Extra information if required}

\newpage{}

\section*{Appendix --- Reflection}

\wss{Not required for CAS 741 projects}

The information in this section will be used to evaluate the team members on the
graduate attribute of Problem Analysis and Design.

\input{../../Reflection.tex}

\begin{enumerate}
  \item What went well while writing this deliverable? 
  \item What pain points did you experience during this deliverable, and how
    did you resolve them?
  \item Which of your design decisions stemmed from speaking to your client(s)
  or a proxy (e.g. your peers, stakeholders, potential users)? For those that
  were not, why, and where did they come from?
  \item While creating the design doc, what parts of your other documents (e.g.
  requirements, hazard analysis, etc), it any, needed to be changed, and why?
  \item What are the limitations of your solution?  Put another way, given
  unlimited resources, what could you do to make the project better? (LO\_ProbSolutions)
  \item Give a brief overview of other design solutions you considered.  What
  are the benefits and tradeoffs of those other designs compared with the chosen
  design?  From all the potential options, why did you select the documented design?
  (LO\_Explores)
\end{enumerate}


\end{document}