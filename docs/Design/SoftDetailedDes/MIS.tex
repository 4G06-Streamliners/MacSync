\documentclass[12pt, titlepage]{article}

\usepackage{amsmath, mathtools}

\usepackage[round]{natbib}
\usepackage{amsfonts}
\usepackage{amssymb}
\usepackage{graphicx}
\usepackage{colortbl}
\usepackage{xr}
\usepackage{hyperref}
\usepackage{longtable}
\usepackage{xfrac}
\usepackage{tabularx}
\usepackage{float}
\usepackage{siunitx}
\usepackage{booktabs}
\usepackage{multirow}
\usepackage[section]{placeins}
\usepackage{caption}
\usepackage{fullpage}

\hypersetup{
bookmarks=true,     % show bookmarks bar?
colorlinks=true,       % false: boxed links; true: colored links
linkcolor=red,          % color of internal links (change box color with linkbordercolor)
citecolor=blue,      % color of links to bibliography
filecolor=magenta,  % color of file links
urlcolor=cyan          % color of external links
}

\usepackage{array}

\externaldocument{../../SRS/SRS}

\input{../../Comments}
%% Common Parts

\newcommand{\progname}{Software Engineering} % PUT YOUR PROGRAM NAME HERE
\newcommand{\authname}{Team \#12, Streamliners
\\ Mahad Ahmed
\\ Abyan Jaigirdar
\\ Prerna Prabhu
\\ Farhan Rahman
\\ Ali Zia} % AUTHOR NAMES                  

\usepackage{hyperref}
    \hypersetup{colorlinks=true, linkcolor=blue, citecolor=blue, filecolor=blue,
                urlcolor=blue, unicode=false}
    \urlstyle{same}     

\begin{document}

\title{Module Interface Specification for \progname{}}

\author{\authname}

\date{\today}

\maketitle

\pagenumbering{roman}

\section{Revision History}

\begin{tabularx}{\textwidth}{p{3cm}p{2cm}X}
\toprule {\bf Date} & {\bf Version} & {\bf Notes}\\
\midrule
Date 1 & 1.0 & Notes\\
Date 2 & 1.1 & Notes\\
\bottomrule
\end{tabularx}

~\newpage

\section{Symbols, Abbreviations and Acronyms}

See SRS Documentation at \wss{give url}

\wss{Also add any additional symbols, abbreviations or acronyms}

\newpage

\tableofcontents

\newpage

\pagenumbering{arabic}

\section{Introduction}

The following document details the Module Interface Specifications for
\wss{Fill in your project name and description}

Complementary documents include the System Requirement Specifications
and Module Guide.  The full documentation and implementation can be
found at \url{...}.  \wss{provide the url for your repo}

\section{Notation}

\wss{You should describe your notation.  You can use what is below as
  a starting point.}

The structure of the MIS for modules comes from \citet{HoffmanAndStrooper1995},
with the addition that template modules have been adapted from
\cite{GhezziEtAl2003}.  The mathematical notation comes from Chapter 3 of
\citet{HoffmanAndStrooper1995}.  For instance, the symbol := is used for a
multiple assignment statement and conditional rules follow the form $(c_1
\Rightarrow r_1 | c_2 \Rightarrow r_2 | ... | c_n \Rightarrow r_n )$.

The following table summarizes the primitive data types used by \progname. 

\begin{center}
\renewcommand{\arraystretch}{1.2}
\noindent 
\begin{tabular}{l l p{7.5cm}} 
\toprule 
\textbf{Data Type} & \textbf{Notation} & \textbf{Description}\\ 
\midrule
character & char & a single symbol or digit\\
integer & $\mathbb{Z}$ & a number without a fractional component in (-$\infty$, $\infty$) \\
natural number & $\mathbb{N}$ & a number without a fractional component in [1, $\infty$) \\
real & $\mathbb{R}$ & any number in (-$\infty$, $\infty$)\\
\bottomrule
\end{tabular} 
\end{center}

\noindent
The specification of \progname \ uses some derived data types: sequences, strings, and
tuples. Sequences are lists filled with elements of the same data type. Strings
are sequences of characters. Tuples contain a list of values, potentially of
different types. In addition, \progname \ uses functions, which
are defined by the data types of their inputs and outputs. Local functions are
described by giving their type signature followed by their specification.

\section{Module Decomposition}

The following table is taken directly from the Module Guide document for this project.

\begin{table}[h!]
\centering
\begin{tabular}{p{0.3\textwidth} p{0.6\textwidth}}
\toprule
\textbf{Level 1} & \textbf{Level 2}\\
\midrule

{Hardware-Hiding} & ~ \\
\midrule

\multirow{7}{0.3\textwidth}{Behaviour-Hiding} & Input Parameters\\
& Output Format\\
& Output Verification\\
& Temperature ODEs\\
& Energy Equations\\ 
& Control Module\\
& Specification Parameters Module\\
\midrule

\multirow{3}{0.3\textwidth}{Software Decision} & {Sequence Data Structure}\\
& ODE Solver\\
& Plotting\\
\bottomrule

\end{tabular}
\caption{Module Hierarchy}
\label{TblMH}
\end{table}

\newpage
~\newpage

\section{MIS of Access Control Module (M5)} 

\subsection{Module}

\texttt{AccessControlService}

\subsection{Uses}

\begin{itemize}
    \item Authorization Module (M8)
    \item Persistence Layer (DB): stores roles, features, capabilities, and user assignments.
    \item Caching Layer: speeds up permission lookups.
\end{itemize}


\subsection{Syntax}

\subsubsection{Exported Constants}

\begin{itemize}
    \item \texttt{POLICY\_SCHEMA : JSON}\\
    Defines valid structure for permission rules.
    \item \texttt{AUTH\_ERROR\_CODE : Int}\\
    Code returned when an API call is forbidden.
\end{itemize}

\subsubsection{Exported Access Programs}

\begin{center}
\begin{tabular}{p{4cm} p{4cm} p{4cm} p{4cm}}
\hline
\textbf{Name} & \textbf{In} & \textbf{Out} & \textbf{Exceptions} \\
\hline
\texttt{checkPermission} & action: Str & authorized: bool & AuthorizationError \\
\texttt{assignRole} & userID: Str, role: Str & success: bool & RoleError \\
\texttt{revokeRole} & userID: Str, role: Str & success: bool & RoleError \\
\texttt{setPolicy} & policy: JSON & success: bool & PolicyError\\
\texttt{getPolicy} & - & policy: JSON & -\\
\hline
\end{tabular}
\end{center}

\subsection{Semantics}

\subsubsection{State Variables}

\begin{itemize}
    \item \texttt{roleAssignments : dict[userID → set[role]]}\\
    Maps each user role to its permitted features.
    \item \texttt{policyDocument : JSON}\\
    Full mapping of roles → features → capabilities
    \item \texttt{policyVersion : Int}\\
    Incremented upon policy update.
\end{itemize}

\subsubsection{Environment Variables}

\begin{itemize}
    \item \texttt{AUTH\_DB\_URI : Str}\\
    Path to policy store.
    \item \texttt{CACHE\_URI : Str}\\
    Cache location for capabilities.
\end{itemize}

\subsubsection{Assumptions}

\begin{itemize}
    \item All callers (Payment, Forms, Check-In, etc.) consult this module before performing protected actions.
    \item M2 calls this module to enforce its authority, it does not make up its own rules.
\end{itemize}

\subsubsection{Access Routine Semantics}

\noindent \texttt{checkPermission(action: Str)}:
\begin{itemize}
\item transition: None
\item input: the attempted action.
\item output: Returns True if the user has the required capability to perform the requested action.
\item exception: \texttt{AuthorizationError}  if the user is not permitted to perform the action.
\end{itemize}

\noindent \texttt{assignRole(userID: Str, role: Str)}:
\begin{itemize}
\item transition: Adds the specified role to the user's stored role assignments. 
\item input: A unique user identifier and a valid role string.
\item output: Returns True on successful role assignment.
\item exception: \texttt{RoleError} if the role does not exist or cannot be assigned.
\end{itemize}

\noindent \texttt{revokeRole(userID: Str, role: Str)}:
\begin{itemize}
\item transition: Removes the specified role from the user's role assignments.
\item input: A unique user identifier and a valid role string.
\item output: Returns True on successful role removal.
\item exception: \texttt{RoleError} if the specified role is not currently assigned to the user.
\end{itemize}

\noindent \texttt{setPolicy(policy: JSON)}:
\begin{itemize}
\item transition: Validates and replaces the policy document with the new policy, increments the internal policy version counter.
\item input: A JSON object containing an updated set of roles, features, capabilities, and guard rules.
\item output: Returns \texttt{True} on success.
\item exception: \texttt{PolicyError} if the policy does not match the defined schema or cannot be persisted.
\end{itemize}

\noindent \texttt{getPolicy()}:
\begin{itemize}
\item transition: None.
\item input: None.
\item output: Returns the current policy document in JSON format.
\item exception: None.
\end{itemize}


\subsubsection{Local Functions}

\begin{itemize}
    \item None.
\end{itemize}

\newpage

\bibliographystyle {plainnat}
\bibliography {../../../refs/References}

\newpage

\section{Appendix} \label{Appendix}

\wss{Extra information if required}

\newpage{}

\section*{Appendix --- Reflection}

\wss{Not required for CAS 741 projects}

The information in this section will be used to evaluate the team members on the
graduate attribute of Problem Analysis and Design.

\input{../../Reflection.tex}

\begin{enumerate}
  \item What went well while writing this deliverable? 
  \item What pain points did you experience during this deliverable, and how
    did you resolve them?
  \item Which of your design decisions stemmed from speaking to your client(s)
  or a proxy (e.g. your peers, stakeholders, potential users)? For those that
  were not, why, and where did they come from?
  \item While creating the design doc, what parts of your other documents (e.g.
  requirements, hazard analysis, etc), it any, needed to be changed, and why?
  \item What are the limitations of your solution?  Put another way, given
  unlimited resources, what could you do to make the project better? (LO\_ProbSolutions)
  \item Give a brief overview of other design solutions you considered.  What
  are the benefits and tradeoffs of those other designs compared with the chosen
  design?  From all the potential options, why did you select the documented design?
  (LO\_Explores)
\end{enumerate}


\end{document}