\documentclass[12pt, titlepage]{article}

\usepackage{amsmath, mathtools}

\usepackage[round]{natbib}
\usepackage{amsfonts}
\usepackage{amssymb}
\usepackage{graphicx}
\usepackage{colortbl}
\usepackage{xr}
\usepackage{hyperref}
\usepackage{longtable}
\usepackage{xfrac}
\usepackage{tabularx}
\usepackage{float}
\usepackage{siunitx}
\usepackage{booktabs}
\usepackage{multirow}
\usepackage[section]{placeins}
\usepackage{caption}
\usepackage{fullpage}

\hypersetup{
bookmarks=true,     % show bookmarks bar?
colorlinks=true,       % false: boxed links; true: colored links
linkcolor=red,          % color of internal links (change box color with linkbordercolor)
citecolor=blue,      % color of links to bibliography
filecolor=magenta,  % color of file links
urlcolor=cyan          % color of external links
}

\usepackage{array}

\externaldocument{../../SRS/SRS}

\input{../../Comments}
%% Common Parts

\newcommand{\progname}{Software Engineering} % PUT YOUR PROGRAM NAME HERE
\newcommand{\authname}{Team \#12, Streamliners
\\ Mahad Ahmed
\\ Abyan Jaigirdar
\\ Prerna Prabhu
\\ Farhan Rahman
\\ Ali Zia} % AUTHOR NAMES                  

\usepackage{hyperref}
    \hypersetup{colorlinks=true, linkcolor=blue, citecolor=blue, filecolor=blue,
                urlcolor=blue, unicode=false}
    \urlstyle{same}     

\begin{document}

\title{Module Interface Specification for \progname{}}

\author{\authname}

\date{\today}

\maketitle

\pagenumbering{roman}

\section{Revision History}

\begin{tabularx}{\textwidth}{p{3cm}p{2cm}X}
\toprule {\bf Date} & {\bf Version} & {\bf Notes}\\
\midrule
Date 1 & 1.0 & Notes\\
Date 2 & 1.1 & Notes\\
\bottomrule
\end{tabularx}

~\newpage

\section{Symbols, Abbreviations and Acronyms}

See SRS Documentation at \href{https://github.com/4G06-Streamliners/MacSync-SRS/blob/main/index.pdf}{SRS} and MG Documentation at \href{https://github.com/4G06-Streamliners/MacSync/blob/main/docs/Design/SoftArchitecture/MG.pdf}{MG}.

\newpage

\tableofcontents

\newpage

\pagenumbering{arabic}

\section{Introduction}

The following document details the Module Interface Specification for the Large Event Management System (LEMS).
It defines the detailed interfaces, inputs, outputs, and dependencies among the software modules that comprise the
system. Building on the design decomposition outlined in the Module Guide and the functional and non-functional
requirements in the Software Requirements Specification (SRS), this document establishes a precise contract for how
each module communicates and interacts within the overall architecture.\\

\noindent
The LEMS platform is a centralized event management solution developed for the McMaster Engineering Society (MES)
to support the organization and execution of large-scale student events such as the Fireball Formal, Graduation
Formal, and Pub Nights. The system integrates event registration, ticketing, waivers, payment processing, and
check-in into a single platform accessible via web and mobile interfaces. To ensure modularity, maintainability, and
reliability, the system is decomposed into independent yet cohesive modules such as the Payment Processing Module,
Role-Based and Feature-Based Access Control (RBAC/FBAC) Module, and the Bus/Table/RSVP Sign-Up Module.\\

\noindent
Complementary documents include the System Requirement Specifications
and Module Guide.  The full documentation and implementation can be
found at \url{https://github.com/4G06-Streamliners/MacSync}.

\section{Notation}

\wss{You should describe your notation.  You can use what is below as
  a starting point.}

The structure of the MIS for modules comes from \citet{HoffmanAndStrooper1995},
with the addition that template modules have been adapted from
\cite{GhezziEtAl2003}.  The mathematical notation comes from Chapter 3 of
\citet{HoffmanAndStrooper1995}.  For instance, the symbol := is used for a
multiple assignment statement and conditional rules follow the form $(c_1
\Rightarrow r_1 | c_2 \Rightarrow r_2 | ... | c_n \Rightarrow r_n )$.

The following table summarizes the primitive data types used by \progname. 

\begin{center}
\renewcommand{\arraystretch}{1.2}
\noindent 
\begin{tabular}{l l p{7.5cm}} 
\toprule 
\textbf{Data Type} & \textbf{Notation} & \textbf{Description}\\ 
\midrule
character & char & a single symbol or digit\\
integer & $\mathbb{Z}$ & a number without a fractional component in (-$\infty$, $\infty$) \\
natural number & $\mathbb{N}$ & a number without a fractional component in [1, $\infty$) \\
real & $\mathbb{R}$ & any number in (-$\infty$, $\infty$)\\
\bottomrule
\end{tabular} 
\end{center}

\noindent
The specification of \progname \ uses some derived data types: sequences, strings, and
tuples. Sequences are lists filled with elements of the same data type. Strings
are sequences of characters. Tuples contain a list of values, potentially of
different types. In addition, \progname \ uses functions, which
are defined by the data types of their inputs and outputs. Local functions are
described by giving their type signature followed by their specification.

\section{Module Decomposition}

The following table is taken directly from the Module Guide document for this project.

\begin{table}[H]
\centering
\begin{tabular}{p{0.3\textwidth} p{0.6\textwidth}}
\toprule
\textbf{Level 1} & \textbf{Level 2}\\
\midrule

{Hardware-Hiding Module} & ~ \\
\midrule

\multirow{8}{0.3\textwidth}{Behaviour-Hiding Module} 
& Payment Processing Module\\
& RBAC/FBAC Access Control Module\\
& Sign-Up Module\\
& \hspace{1em} Sign-Up Adapter\\
& \hspace{1em} Bus Sign-Up Module\\
& \hspace{1em} Table Sign-Up Module\\
& \hspace{1em} RSVP Sign-Up Module\\
& User Authorization Module\\
\midrule

\multirow{4}{0.3\textwidth}{Software Decision Module}
& Payment Configuration Module\\
& Access Control Module\\
& Registration Rules Module\\
& Notification Handling Module\\
\bottomrule
\end{tabular}
\caption{Module Hierarchy}
\label{TblMH}
\end{table}

\newpage
~\newpage


\newpage
\section{Module Interface Specifications} \label{SecMIS}

This section provides the Module Interface Specifications (MIS) for each module 
defined in the Module Guide. Each module is documented independently, following 
the notation and structure described in Section 2.

\subsection{MIS of Payment Processing Module (M1)} \label{MIS_M1}

\subsubsection{Module}

Payment Processing Module (M1)

This module prepares and submits payment requests from the client side.  
It validates the payment data, attaches authentication headers, and communicates 
with the backend API that integrates with Stripe. The module does not handle 
any payment verification locally. Instead, verification and 
execution of charges or refunds are performed by the backend using Stripe’s 
secure payment APIs, configured in the Payment Configuration Module (M4).

\subsubsection{Uses}

\begin{itemize}
    \item API Layer (HTTP communication with backend)
    \item User Authorization Module (M8)
    \item Payment Configuration Module (M4)
    \item Notification Handling Module (M7)
\end{itemize}

\subsection{Syntax}

\subsubsection{Exported Constants}
N/A

\subsubsection{Exported Access Programs}

\begin{center}
\renewcommand{\arraystretch}{1.2}
\begin{tabularx}{\textwidth}{l l l X}
\toprule
\textbf{Name} & \textbf{In} & \textbf{Out} & \textbf{Exceptions} \\
\midrule
submitPayment       & PaymentInfo & ConfirmationResponse & InvalidPaymentData, NetworkError \\
validatePaymentData & PaymentInfo & Boolean              & InvalidPaymentData \\
requestRefund       & RefundRequest & RefundResponse      & InvalidRefundRequest, NetworkError \\
\bottomrule
\end{tabularx}
\end{center}

\subsection{Semantics}

\subsubsection{State Variables}

N/A

\subsubsection{Environment Variables}

\begin{itemize}
    \item Network connection to the backend API
    \item AuthToken from User Authorization Module (M8)
\end{itemize}

\subsubsection{Assumptions}

\begin{itemize}
    \item User must be authenticated before submitting a payment or requesting a refund.
    \item Device must have a stable network connection.
    \item The backend exposes REST endpoints:
        \begin{itemize}
            \item \texttt{POST /api/payments}: initialize Stripe PaymentIntent
            \item \texttt{POST /api/payments/refund}: request a Stripe refund
        \end{itemize}
    \item The backend manages all Stripe secret keys and secure payment operations.
\end{itemize}

\subsubsection{Access Routine Semantics}

\noindent \textbf{submitPayment}(paymentInfo):

\begin{itemize}
    \item transition: N/A.
    \item output: A ConfirmationResponse containing the client secret, payment result, 
    or any backend-supplied status (e.g., succeeded, failed).
    \item exception: 
        \begin{itemize}
            \item InvalidPaymentData: required fields missing or malformed  
            \item NetworkError: request fails or cannot reach backend 
        \end{itemize}
\end{itemize}

\vspace{0.2cm}

\noindent \textbf{validatePaymentData}(paymentInfo):

\begin{itemize}
    \item transition: N/A
    \item output: Returns true if all required fields (amount, event ID, billing details) 
    are correctly defined.
    \item exception: InvalidPaymentData if fields are empty or violate basic constraints.
\end{itemize}

\vspace{0.2cm}

\noindent \textbf{requestRefund}(refundRequest):

\begin{itemize}
    \item transition: N/A
    \item output: A RefundResponse containing refund status, amount refunded,  
    and backend-supplied outcome (e.g. pending, succeeded, failed).
    \item exception:
        \begin{itemize}
            \item InvalidRefundRequest: missing payment ID, invalid amount  
            \item NetworkError: unable to send or receive the refund request
        \end{itemize}
\end{itemize}

\subsubsection{Local Functions}

\begin{itemize}
    \item submitAPIRequest(jsonBody, authToken, endpoint): sends the HTTP POST request to the backend API.
    \item handleAPIResponse(response): converts backend JSON into the correct response object 
          (ConfirmationResponse or RefundResponse).
\end{itemize}

\subsection{MIS of RBAC/FBAC Access Control Module (M2)} 

\subsubsection{Module}

\texttt{RBACModule}

\subsubsection{Uses}

\begin{itemize}
    \item User Authorization Module (M8)
    \item Access Control Service Module (M5)
\end{itemize}

\subsubsection{Syntax}

\subsubsection{Exported Constants}

\begin{itemize}
    \item \texttt{DEFAULT\_ROLE}: \texttt{str}\\
    Default user role when signing up for the platform.
    \item \texttt{ACCESS\_POLICY\_SCHEMA: JSON}\\
    Schema for the capability snapshot retrieved from backend M5.
\end{itemize}

\subsubsection{Exported Access Programs}

\begin{center}
\begin{tabular}{p{4cm} p{4cm} p{4cm} p{4cm}}
\hline
\textbf{Name} & \textbf{In} & \textbf{Out} & \textbf{Exceptions} \\
\hline
\texttt{getUserRole} & userID: Str & role: Str & UserNotFound \\
\texttt{getUserCapabilities} & userID: Str & capabilities: list[Str] & UserNotFound \\
\texttt{authorizeAction} & action: Str & authorized: bool & AuthorizationError\\
\texttt{syncCapabilities} & userID: Str & success: bool & FetchError \\
\hline
\end{tabular}
\end{center}

\subsection{Semantics}

\subsubsection{State Variables}

\begin{itemize}
    \item \texttt{capabilitySnapshot : dict[Str, list[Str]]}\\
    Locally cached copy of backend-authorized capabilities for the active user.
    \item \texttt{userRole: Str}\\
    UI-level role used for UX grouping.
\end{itemize}

\subsubsection{Environment Variables}

\begin{itemize}
    \item \texttt{ACCESS\_CONTROL\_ENDPOINT : Str}\\
    Backend API to fetch the authoritative permissions.
\end{itemize}

\subsubsection{Assumptions}

\begin{itemize}
    \item Authorization requests are made only for defined features or actions.
    \item M5 always provides the source-of-truth permission set.
\end{itemize}

\subsubsection{Access Routine Semantics}

\noindent \texttt{getUserRole(userID: Str)}:
\begin{itemize}
\item transition: None
\item input: A unique user identifier.
\item output: Returns the role string associated with the given user.
\item exception: \texttt{UserNotFound} If UUID cannot be identified.
\end{itemize}

\noindent \texttt{getUserCapabilities(userID: Str)}:
\begin{itemize}
\item transition: None
\item input: A unique user identifier.
\item output: Returns list of available features/actions.
\item exception: \texttt{UserNotFound} If UUID cannot be identified.
\end{itemize}

\noindent \texttt{authorizeAction(userID: Str, action: Str)}:
\begin{itemize}
\item transition: None
\item input: User ID and attempted action.
\item output: Returns \texttt{True} if the action is permitted by user's role/capabilities.
\item exception: \texttt{AuthorizationError}  if access is denied.
\end{itemize}

\noindent \texttt{syncCapabilities(userID: Str)}:
\begin{itemize}
\item transition: Fetches the capability snapshot from M5 and updates internal policy schema.
\item input: User ID
\item output: Returns \texttt{True} on success.
\item exception: \texttt{FetchError} if backend cannot be reached.
\end{itemize}

\subsubsection{Local Functions}

\begin{itemize}
    \item None.
\end{itemize}

\subsection{MIS of Sign-Up Module (M3)} \label{MIS_M3}

\subsubsection{Module}

\wss{SignupModule}

\subsubsection{Uses}
\begin{itemize}
    \item A back-end or configuration service to store and retrieve confirmed signups.
    \item A front-end/UI component for users to input sign-up information.
\end{itemize}

\subsubsection{Syntax}

\subsubsection{Exported Constants}
\begin{itemize}
    \item \texttt{SIGNUP\_TYPES}: ("RSVP", "Table", "Bus")
    \begin{itemize}
        \item Supported types of signup modules.
    \end{itemize}

    \item \texttt{MAX\_PARTICIPANTS\_PER\_EVENT}: $\mathbb{N}$
    \begin{itemize}
        \item Maximum number of participants allowed per event.
    \end{itemize}
\end{itemize}

\subsubsection{Exported Access Programs}
\begin{center}
\begin{tabular}{p{2cm} p{4cm} p{4cm} p{2cm}}
\hline
\textbf{Name} & \textbf{In} & \textbf{Out} & \textbf{Exceptions} \\
\hline
submitSignup & userID: Str \newline eventID: Str \newline data: JSON & success: bool & SignupError \\
\hline
getSignupStatus & userID: Str \newline eventID: Str & status: JSON & NotFoundError \\
\hline
cancelSignup & userID: Str \newline eventID: Str & success: bool & CancellationError \\
\hline
\end{tabular}
\end{center}

\subsection{Semantics}

\subsubsection{State Variables}
\begin{itemize}
    \item signupDetails: dict[(userID, eventID): JSON]
    \begin{itemize}
        \item Stores all signup information and metadata.
    \end{itemize}
\end{itemize}

\subsubsection{Environment Variables}
\begin{itemize}
    \item SIGNUP\_API: Str
    \begin{itemize}
        \item API endpoint to database for sending and storing signup data to backend.
    \end{itemize}
\end{itemize}

\subsubsection{Assumptions}
\begin{itemize}
    \item A user has a valid session where they are authenticated before accessing the signup module.
\end{itemize}

\subsubsection{Access Routine Semantics}

\noindent submitSignup(UserID, eventID, data):
\begin{itemize}
\item transition: Adds new entry to signupDetails for given userID and eventID.
\item output: Returns true if signup is successful.
\item exception: SignupError if validation fails or event is full.
\end{itemize}

\noindent getSignupStatus(userID, eventID):
\begin{itemize}
\item transition: None.
\item output: Returns JSON information with users signup details for the given event.
\item exception: NotFoundError if no record exists.
\end{itemize}

\noindent cancelSignup(userID, eventID):
\begin{itemize}
\item transition: Removes signup for given userID and eventID.
\item output: Returns true on successful deletion.
\item exception: CancellationError if no record is found for deletion
\end{itemize}

\subsubsection{Local Functions}

\noindent validateSignupData(formData: JSON):
\begin{itemize}
\item transition: None.
\item output: Returns true if all required fields are complete.
\item exception: ValidationError if required fields are missing or have invalid data format.s
\end{itemize}

\noindent checkCapacity(eventID: str):
\begin{itemize}
\item transition: None.
\item output: Returns true if the current signup added to total signups is < MAX\_PARTICIPANTS.
\item exception: CapacityExceededError if event has reached full capacity.
\end{itemize}

\subsection{MIS of Payment Configuration Module (M4)}

\subsubsection{Module}

\texttt{PaymentConfigurationModule}\\

\noindent
This module centralizes all configuration required for secure payment operations.
It exposes safe, restricted payment settings to other modules such as the Payment
Processing Module (M1) while hiding sensitive Stripe configuration logic and
backend API wiring. It ensures a single source of truth for currency rules,
tax configuration, receipt formatting, and API endpoint definitions related to
payments.

\subsubsection{Uses}

\begin{itemize}
    \item Payment Processing Module (M1): Uses exported configuration values
    to construct valid payment requests and interact with backend payment endpoints.
    \item User Authorization Module (M8): Provides authentication tokens required
    when modules call backend payment APIs.
    \item Access Control Module (M5): Ensures that only authorized user roles may
    access payment configuration or initiate restricted payment operations.
    \item Notification Handling Module (M7): Uses configured receipt and invoice
    formatting rules when sending payment confirmations.
    \item Backend Stripe Integration Service: Provides the actual
    PaymentIntent creation, confirmation, and refund endpoints (e.g., 
    \texttt{/api/payments}, \texttt{/api/payments/refund}).
\end{itemize}

\subsubsection{Syntax}

\subsubsection{Exported Constants}

\begin{itemize}
    \item \texttt{STRIPE\_PUBLIC\_KEY : Str}\\
    The publishable Stripe key used by the frontend; never contains secret keys.
    \item \texttt{PAYMENT\_API\_BASE : URL}\\
    Base backend endpoint for payment operations.
    \item \texttt{CURRENCY\_CODE : Str}\\
    Currency used for all events (\texttt{CAD}).
    \item \texttt{TAX\_RATE : Float}\\
    The global tax percentage applied to event payments.
    \item \texttt{INTENT\_CACHE\_TTL\_SEC : Nat}\\
    TTL for caching Stripe PaymentIntent client secrets.
\end{itemize}

\subsubsection{Exported Access Programs}

\begin{center}
\renewcommand{\arraystretch}{1.2}
\begin{tabularx}{\textwidth}{l l l X}
\toprule
\textbf{Name} & \textbf{In} & \textbf{Out} & \textbf{Exceptions} \\
\midrule
\texttt{getPublicStripeKey} & -- & Str & KeyNotAvailable \\
\texttt{getPaymentAPIEndpoint} & OperationType & URL & InvalidOperationType \\
\texttt{getCurrencyConfig} & -- & CurrencyConfig & -- \\
\texttt{shouldRefreshPaymentIntent} & LastUpdated: Time & Bool & -- \\
\texttt{getReceiptConfig} & -- & ReceiptConfig & -- \\
\bottomrule
\end{tabularx}
\end{center}

\subsubsection{Semantics}

\subsubsection{State Variables}

\begin{itemize}
    \item \texttt{publicKey : Str}
    \item \texttt{paymentAPIMap : dict[OperationType $\rightarrow$ URL]}
    \item \texttt{currencyConfig : CurrencyConfig}
    \item \texttt{receiptConfig : ReceiptConfig}
    \item \texttt{cacheTTL : Nat}
\end{itemize}

\subsubsection{Environment Variables}

\begin{itemize}
    \item \texttt{ENV\_STRIPE\_PUBLIC\_KEY : Str} - Loaded from build-time environment.
    \item \texttt{ENV\_PAYMENT\_API\_BASE : URL} - Backend service base URL.
    \item \texttt{SystemTime : Time} - Used for TTL and expiration checks.
\end{itemize}

\subsubsection{Assumptions}

\begin{itemize}
    \item Stripe secret keys are stored only on the backend and never exposed to the client.
    \item Backend exposes the routes \texttt{/api/payments}, 
    \texttt{/api/payments/confirm}\allowbreak, and \texttt{/api/payments/refund}.
    \item All callers of this module have been authenticated by M8 and authorized by M5.
    \item Configuration is loaded once per application session.
\end{itemize}

\subsubsection{Access Routine Semantics}

\noindent\textbf{\texttt{getPublicStripeKey()}}
\begin{itemize}
\item transition: none
\item output: \texttt{publicKey}
\item exception: \texttt{KeyNotAvailable} if key is missing or invalid.
\end{itemize}

\vspace{0.2cm}

\noindent\textbf{\texttt{getPaymentAPIEndpoint(opType)}}
\begin{itemize}
\item transition: none
\item output: returns URL mapped to the given operation type.
\item exception: \texttt{InvalidOperationType} if no mapping is defined.
\end{itemize}

\vspace{0.2cm}

\noindent\textbf{\texttt{getCurrencyConfig()}}
\begin{itemize}
\item transition: none
\item output: returns currency and tax configuration.
\item exception: none.
\end{itemize}

\vspace{0.2cm}

\noindent\textbf{\texttt{shouldRefreshPaymentIntent(lastUpdated)}}
\begin{itemize}
\item transition: none
\item output: \texttt{true} if the PaymentIntent cache TTL has expired.
\item exception: none.
\end{itemize}

\vspace{0.2cm}

\noindent\textbf{\texttt{getReceiptConfig()}}
\begin{itemize}
\item transition: none
\item output: returns canonical receipt formatting settings for notifications.
\item exception: none.
\end{itemize}

\subsubsection{Local Functions}

\begin{itemize}
    \item \texttt{loadEnvValue(key)}: safely retrieves build-time environment variables.
    \item \texttt{buildEndpoint(base, suffix)}: constructs consistent backend payment URLs.
    \item \texttt{validatePublicKey(str)}: checks prefix/type of Stripe public key.
    \item \texttt{computeTTL(t)}: helper for determining TTL expiration.
\end{itemize}

\subsection{MIS of Access Control Module (M5)}

\subsubsection{Module}

\texttt{AccessControlService}

\subsubsection{Uses}

\begin{itemize}
    \item Authorization Module (M8)
    \item Persistence Layer (DB): stores roles, features, capabilities, and user assignments.
    \item Caching Layer: speeds up permission lookups.
\end{itemize}

\subsubsection{Syntax}

\subsubsection{Exported Constants}

\begin{itemize}
    \item \texttt{POLICY\_SCHEMA : JSON}\\
    Defines valid structure for permission rules.
    \item \texttt{AUTH\_ERROR\_CODE : Int}\\
    Code returned when an API call is forbidden.
\end{itemize}

\subsubsection{Exported Access Programs}

\begin{center}
\begin{tabular}{p{4cm} p{4cm} p{4cm} p{4cm}}
\hline
\textbf{Name} & \textbf{In} & \textbf{Out} & \textbf{Exceptions} \\
\hline
\texttt{checkPermission} & action: Str & authorized: bool & AuthorizationError \\
\texttt{assignRole} & userID: Str, role: Str & success: bool & RoleError \\
\texttt{revokeRole} & userID: Str, role: Str & success: bool & RoleError \\
\texttt{setPolicy} & policy: JSON & success: bool & PolicyError \\
\texttt{getPolicy} & - & policy: JSON & - \\
\hline
\end{tabular}
\end{center}

\subsubsection{Semantics}

\subsubsection{State Variables}

\begin{itemize}
    \item \texttt{roleAssignments : dict[userID → set[role]]}\\
    Maps each user role to its permitted features.
    \item \texttt{policyDocument : JSON}\\
    Full mapping of roles → features → capabilities
    \item \texttt{policyVersion : Int}\\
    Incremented upon policy update.
\end{itemize}

\subsubsection{Environment Variables}

\begin{itemize}
    \item \texttt{AUTH\_DB\_URI : Str}\\
    Path to policy store.
    \item \texttt{CACHE\_URI : Str}\\
    Cache location for capabilities.
\end{itemize}

\subsubsection{Assumptions}

\begin{itemize}
    \item All callers (Payment, Forms, Check-In, etc.) consult this module before performing protected actions.
    \item M2 calls this module to enforce its authority, it does not make up its own rules.
\end{itemize}

\subsubsection{Access Routine Semantics}

\noindent \texttt{checkPermission(action: Str)}:
\begin{itemize}
\item transition: None
\item input: the attempted action.
\item output: Returns True if the user has the required capability to perform the requested action.
\item exception: \texttt{AuthorizationError} if the user is not permitted to perform the action.
\end{itemize}

\noindent \texttt{assignRole(userID: Str, role: Str)}:
\begin{itemize}
\item transition: Adds the specified role to the user's stored role assignments. 
\item input: A unique user identifier and a valid role string.
\item output: Returns True on successful role assignment.
\item exception: \texttt{RoleError} if the role does not exist or cannot be assigned.
\end{itemize}

\noindent \texttt{revokeRole(userID: Str, role: Str)}:
\begin{itemize}
\item transition: Removes the specified role from the user's role assignments.
\item input: A unique user identifier and a valid role string.
\item output: Returns True on successful role removal.
\item exception: \texttt{RoleError} if the specified role is not currently assigned to the user.
\end{itemize}

\noindent \texttt{setPolicy(policy: JSON)}:
\begin{itemize}
\item transition: Validates and replaces the policy document with the new policy, increments the internal policy version counter.
\item input: A JSON object containing an updated set of roles, features, capabilities, and guard rules.
\item output: Returns \texttt{True} on success.
\item exception: \texttt{PolicyError} if the policy does not match the defined schema or cannot be persisted.
\end{itemize}

\noindent \texttt{getPolicy()}:
\begin{itemize}
\item transition: None.
\item input: None.
\item output: Returns the current policy document in JSON format.
\item exception: None.
\end{itemize}

\subsubsection{Local Functions}

\begin{itemize}
    \item None.
\end{itemize}

\subsection{MIS of Registration Rules Module (M6)} \label{MIS_M6}

\subsubsection{Module}

Registration Rules Module (M6)

This module enforces all event-level registration constraints for 
Bus Sign-Up (M3.1), Table Sign-Up (M3.2), and RSVP Sign-Up (M3.3). It retrieves 
event rules such as capacity limits, table constraints, bus seating limits, 
registration deadlines, and RSVP policies from the backend datastore.  
It validates whether a user is allowed to register, whether an event has 
remaining capacity, and whether any additional restrictions apply.  
All rule enforcement occurs before entries are recorded or notifications are triggered.

\subsubsection{Uses}

\begin{itemize}
    \item Access Control Module (M5): for checking user roles and eligibility
    \item User Authorization Module (M8): for identity data required in rule checks
    \item Backend datastore (via data access layer): for retrieving event rules and counts
    \item Sign-Up Module (M3): requests rule validation during sign-up flows
\end{itemize}

\subsection{Syntax}

\subsubsection{Exported Constants}

N/A

\subsubsection{Exported Access Programs}

\begin{center}
\renewcommand{\arraystretch}{1.2}
\begin{tabularx}{\textwidth}{l l l X}
\toprule
\textbf{Name} & \textbf{In} & \textbf{Out} & \textbf{Exceptions} \\
\midrule
validateCapacity     & EventInfo, UserInfo & Boolean & EventFull, InvalidEvent \\
validateDeadline     & EventInfo, UserInfo & Boolean & DeadlinePassed, InvalidEvent \\
validateEligibility  & EventInfo, UserInfo & Boolean & PermissionDenied \\
getEventRules        & EventID             & EventRules & InvalidEvent \\
\bottomrule
\end{tabularx}
\end{center}

\subsection{Semantics}

\subsubsection{State Variables}

None.

\subsubsection{Environment Variables}

\begin{itemize}
    \item Database connection (event metadata, capacity counts, deadlines)
\end{itemize}

\subsubsection{Assumptions}

\begin{itemize}
    \item The database contains valid and up-to-date event policies.
    \item EventInfo passed in from M3 contains a valid event ID.
    \item UserInfo contains a valid authenticated user ID.
\end{itemize}

\subsubsection{Access Routine Semantics}

\noindent \textbf{validateCapacity}(eventInfo, userInfo):

\begin{itemize}
    \item transition: N/A
    \item output: Returns true if event capacity has not been reached.
    \item exception: EventFull if capacity is exhausted, InvalidEvent if event does not exist.
\end{itemize}

\vspace{0.15cm}

\noindent \textbf{validateDeadline}(eventInfo, userInfo):

\begin{itemize}
    \item transition: N/A
    \item output: Returns true if registration is still open.
    \item exception: DeadlinePassed if the cutoff time has passed, InvalidEvent otherwise.
\end{itemize}

\vspace{0.15cm}

\noindent \textbf{validateEligibility}(eventInfo, userInfo):

\begin{itemize}
    \item transition: N/A
    \item output: Returns true if the user meets role-based or event-specific criteria.
    \item exception: PermissionDenied if the user is not allowed to register.
\end{itemize}

\vspace{0.15cm}

\noindent \textbf{getEventRules}(eventID):

\begin{itemize}
    \item transition: N/A
    \item output: Returns the complete rule set for the given event (capacity, deadlines, policies).
    \item exception: InvalidEvent if the event does not exist.
\end{itemize}

\subsubsection{Local Functions}

\begin{itemize}
    \item lookupRules(eventID): fetches the rule set for an event from the datastore.
    \item isBeforeDeadline(eventInfo): compares current time to stored deadline.
    \item hasRemainingCapacity(eventInfo): queries current participant count.
    \item meetsEligibilityCriteria(eventInfo, userInfo): checks role access and custom restrictions.
\end{itemize}

\subsection{MIS of Notification Handling Module (M7)} \label{MIS_M7}

\subsubsection{Module}

NotificationModule

\subsubsection{Uses}
\begin{itemize}
    \item A front-end/UI component for displaying notifications.
\end{itemize}

\subsection{Syntax}

\subsubsection{Exported Constants}
\begin{itemize}
    \item \texttt{NOTIFICATION\_TYPES}: 
    \begin{itemize}
        \item ("Confirmation", "Reminder", "Cancellation")
        \item Supported types of notifications.
    \end{itemize}
\end{itemize}

\subsubsection{Exported Access Programs}

\begin{center}
\begin{tabular}{p{2.5cm} p{4.5cm} p{3.5cm} p{3cm}}
\hline
\textbf{Name} & \textbf{In} & \textbf{Out} & \textbf{Exceptions} \\
\hline
SendNotification 
& \texttt{userID: Str}\\ \texttt{message: Str}\\ \texttt{notification\_type: Str}
& success: bool 
& NotificationError \\
\hline
getNotificationHistory 
& \texttt{userID: Str} 
& Notifications: List[Str] 
& RetrievalError \\
\hline
subscribeUser 
& \texttt{userID: Str}\\ \texttt{subscribed: bool}
& success: bool 
& SubscribingError \\
\hline
UnsubscribeUser 
& \texttt{userID: Str}\\ \texttt{subscribed: bool}
& success: bool 
& \texttt{UnsubscribingError} \\
\hline
\end{tabular}
\end{center}

\subsection{Semantics}

\subsubsection{State Variables}
\begin{itemize}
    \item userNotifPreference: dict[UserID: bool]
    \begin{itemize}
        \item Contains users notification preference.
    \end{itemize}
\end{itemize}

\subsubsection{Environment Variables}
\begin{itemize}
    \item \texttt{EMAIL\_API}: Str
    \begin{itemize}
        \item API endpoint for email notification delivery.
    \end{itemize}
    \item \texttt{SMS\_API}: Str
    \begin{itemize}
        \item API endpoint for SMS notification delivery.
    \end{itemize}
\end{itemize}

\subsubsection{Assumptions}
\begin{itemize}
    \item Users have opted into receiving notifications.
    \item Third party email and SMS APIs are functional.
\end{itemize}

\subsubsection{Access Routine Semantics}

\noindent SendNotification(userID, message, notification\_type):
\begin{itemize}
\item transition: send notification via Email and SMS API.
\item output: Returns true if successfully sent.
\item exception: NotificationError if API fails and message cannot be sent.
\end{itemize}

\noindent getNotificationHistory(userID):
\begin{itemize}
\item transition: None.
\item output: Returns list of previously sent out notifications.
\item exception: RetrievalError if no history exists.
\end{itemize}

\noindent subscribeUser(userID, subscribed):
\begin{itemize}
\item transition: Adds user to the userNotifPreference list and sets value to true.
\item output: Returns true upon successful subscription.
\item exception: SubscribingError if invalid subscription.
\end{itemize}

\noindent UnsubscribeUser(userID, subscribed):
\begin{itemize}
\item transition: Updates userNotifPreference of specified user to false.
\item output: Returns true on successful unsubscribing.
\item exception: UnsubscribingError if user is not subscribed.
\end{itemize}

\subsubsection{Local Functions}
\begin{itemize}
\item None
\end{itemize}

\subsection{MIS of User Authorization Module (M8)} \label{MIS_M8}

\subsubsection{Module}

\texttt{UserAuthorizationModule}\\

\noindent
This module provides authentication and identity verification for all other 
modules in the system. It manages login sessions, token issuance, token 
validation, and user identity retrieval. The module integrates with external 
authentication providers when applicable and acts as the system’s source of 
truth for authenticated user identity data used by RBAC/FBAC (M5/M2) and any 
module requiring secure access.

\subsubsection{Uses}

\begin{itemize}
    \item Access Control Module (M5): Uses identity and token validation functions to determine a user's permissions.
    \item Payment Processing Module (M1): Requires a verified user identity before initiating payment or refund operations.
    \item Sign-Up Module (M3): Uses authenticated user identifiers when creating bus/table/RSVP sign-up records.
    \item Notification Handling Module (M7): Uses authenticated user information to route notifications to the correct recipient.
    \item External Authentication Provider (e.g., OAuth/SSO): Provides the 
    underlying authentication mechanism for secure login.
\end{itemize}

\subsection{Syntax}

\subsubsection{Exported Constants}

\begin{itemize}
    \item \texttt{TOKEN\_EXPIRY\_MIN : Nat}\\
    The number of minutes before an issued access token expires.
    \item \texttt{REFRESH\_TOKEN\_SUPPORTED : Bool}\\
    Indicates whether refresh tokens are enabled in the current 
    deployment configuration.
\end{itemize}

\subsubsection{Exported Access Programs}

\begin{center}
\renewcommand{\arraystretch}{1.2}
\begin{tabularx}{\textwidth}{l l l X}
\toprule
\textbf{Name} & \textbf{In} & \textbf{Out} & \textbf{Exceptions} \\
\midrule
\texttt{login} & Credentials & AuthToken & InvalidCredentials, NetworkError \\
\texttt{logout} & AuthToken & Bool & TokenError \\
\texttt{validateToken} & AuthToken & Bool & TokenExpired, TokenInvalid \\
\texttt{getUserInfo} & AuthToken & UserProfile & TokenInvalid, NetworkError \\
\texttt{refreshToken} & RefreshToken & AuthToken & TokenInvalid, NotSupported \\
\bottomrule
\end{tabularx}
\end{center}

\subsection{Semantics}

\subsubsection{State Variables}

\begin{itemize}
    \item \texttt{activeSessions : dict[token $\rightarrow$ UserID]} \\
    Stores active session mappings for all authenticated users.
    \item \texttt{tokenExpiryTimes : dict[token $\rightarrow$ Time]} \\
    Tracks expiration timestamps for each active token.
\end{itemize}

\subsubsection{Environment Variables}

\begin{itemize}
    \item \texttt{AUTH\_SERVER\_URL : URL} \\
    External authentication provider endpoint (OAuth/SSO).
    \item \texttt{SystemTime : Time} \\
    Used to validate token expiration.
\end{itemize}

\subsubsection{Assumptions}

\begin{itemize}
    \item All modules calling this one must supply a token obtained from \texttt{login} or \texttt{refreshToken}.
    \item The external authentication provider (OAuth/SSO) is available and reliable.
    \item Tokens are cryptographically signed by the authentication provider and cannot be tampered with by the client.
    \item Session expiry rules follow the SRS and Hazard Analysis requirements (e.g., 15-minute admin timeout).
\end{itemize}

\subsubsection{Access Routine Semantics}

\noindent\textbf{\texttt{login(credentials)}}
\begin{itemize}
\item transition: Adds new entry to \texttt{activeSessions} and \texttt{tokenExpiryTimes}.
\item output: Returns a newly issued \texttt{AuthToken}.
\item exception: \texttt{InvalidCredentials} if credentials rejected or \texttt{NetworkError} if authentication provider is unreachable.
\end{itemize}

\vspace{0.2cm}

\noindent\textbf{\texttt{logout(token)}}
\begin{itemize}
\item transition: Removes \texttt{token} from all session maps.
\item output: \texttt{true} on success.
\item exception: \texttt{TokenError} if token is not active.
\end{itemize}

\vspace{0.2cm}

\noindent\textbf{\texttt{validateToken(token)}}
\begin{itemize}
\item transition: none.
\item output: \texttt{true} if token is valid and unexpired.
\item exception: \texttt{TokenInvalid} or \texttt{TokenExpired}.
\end{itemize}

\vspace{0.2cm}

\noindent\textbf{\texttt{getUserInfo(token)}}
\begin{itemize}
\item transition: none.
\item output: Returns the authenticated user's profile (ID, email, name, role).
\item exception: \texttt{TokenInvalid} or \texttt{NetworkError}.
\end{itemize}

\vspace{0.2cm}

\noindent\textbf{\texttt{refreshToken(refreshToken)}}
\begin{itemize}
\item transition: Replaces expired/expiring token with new token, if supported.
\item output: Returns newly issued \texttt{AuthToken}.
\item exception: \texttt{TokenInvalid} if the supplied refresh token is invalid, \texttt{NotSupported} if refresh tokens are disabled.
\end{itemize}

\subsubsection{Local Functions}

\begin{itemize}
    \item \texttt{decodeToken(token)}: Verifies token signature and extracts identity data.
    \item \texttt{isExpired(token)}: Checks whether current time exceeds stored expiry.
    \item \texttt{requestAuthServer(endpoint, payload)}: Handles secure communication with the external authentication provider.
\end{itemize}


\newpage

\bibliographystyle {plainnat}
\bibliography {../../../refs/References}

\newpage

\section{Appendix} \label{Appendix}

\wss{Extra information if required}

\newpage{}

\section*{Appendix --- Reflection}

\wss{Not required for CAS 741 projects}

The information in this section will be used to evaluate the team members on the
graduate attribute of Problem Analysis and Design.

\input{../../Reflection.tex}

\subsection{Mahad Ahmed}

\begin{enumerate}
    \item What went well while writing this deliverable? 

    Something that went well while writing this deliverable was my knowledge for designing system architecture from previous courses. Specifically, 3A04 where we designed system architecture similar to this before creating our software. Using that previous knowledge really helped me complete this document.
    
    \item What pain points did you experience during this deliverable, and how
    did you resolve them?

    I was tasked with creating the MIS for the RBAC Module. I found it challenging due to my lack of understanding of how a RBAC/FBAC system functions, so creating the functions and design for it proved to be challenging. To resolve this, I did research on RBAC/FBAC on how they are designed and their capabilities.
    
\end{enumerate}

\subsection{Prerna Prabhu}

\begin{enumerate}
    \item What went well while writing this deliverable? 

    During this deliverable, I think we learned from our pervious deliverables and we were able to divide up tasks pretty evenly and quickly at the start based on the estimated amount of work each one would be, and we also looked at any dependencies between the different sections prior to assigning the work to each team member. In this case this meant figuring out the modules that we needed for our project, discussing them, and reviewing sections and work that were solely focused on explaining the modules, so that the other parts that depended on module definition could be completed on time without rush.

    \item What pain points did you experience during this deliverable, and how
    did you resolve them?

    During this deliverable, we initially experienced difficulties with what modules we wanted to define and how the would be defined. We were also confused as a team on what some of the requirements were for these modules and how they would be defined. Through discussion as a team, we were able to identify the modules that were needed and tweak any changes and improvements early on, without waiting until closer to the deadline. We were also able to ask more questions to our supervisor and TA to work through these challenges and gain more insight on what was expected from us.
    
\end{enumerate}

\subsection{Ali Zia}

\begin{enumerate}
    \item What went well while writing this deliverable? 

    The division of labour went really well where tasks were assigned fairly and we continued to improve in terms of communicaiton and coordinating our efforts to achieve results and complete our tasks in time

    \item What pain points did you experience during this deliverable, and how
    did you resolve them?

    A challenge we encountered was figuring out how to group the system functionaliteis into modules without overlapping responsiblites and there were moments where team members had slightly different interpretations. However, through open communication and spending time understanding our perspectives and feedback we were able to overcome and address the hallenges

\end{enumerate}

\subsection{Farhan Rahman}

\begin{enumerate}
    \item What went well while writing this deliverable? 

    Our team divided the work effectively and continued the same collaboration approach from earlier milestones, but this time we were more intentional about distributing effort based on each person’s strengths rather than just splitting sections. This helped us produce higher-quality work and kept progress steady. We also leveraged each member’s area of expertise to ensure each part was written clearly and aligned with the system’s design. 

    \item What pain points did you experience during this deliverable, and how
    did you resolve them?

    Because this deliverable involved defining key modules and system components, it took time to fine-tune the details and ensure accuracy that could be carried forward to implementation. Some sections required multiple iterations to make sure they were both technically correct and consistent with the overall system architecture. Coordinating these interdependent parts and aligning them with previous milestones was challenging, but careful reviews and group collaboration helped us resolve it.

\end{enumerate}

\subsection{Abyan Jaigirdar}

\begin{enumerate}
    \item What went well while writing this deliverable? 

    Our team worked really well together during this milestone. We divided up the sections early on and made sure everyone understood their part before starting. Communication was smooth, and we checked in often to make sure everything fit together. This helped us stay organized and finish the deliverable on time without last-minute stress.

    \item What pain points did you experience during this deliverable, and how
    did you resolve them?

    At first, we had trouble deciding how to separate the system into modules and which features belonged where. Some overlap and confusion came up during drafting, but after a few team discussions and review sessions, we cleared things up and finalized a structure we were all confident in.

\end{enumerate}

\subsection{Team}

\begin{enumerate}
  \item Which of your design decisions stemmed from speaking to your client(s)
  or a proxy (e.g. your peers, stakeholders, potential users)? For those that
  were not, why, and where did they come from?

    Several of our design decisions came from directly talking to MES executives. For example, RBAC/FBAC came from stakeholder feedback emphasizing the need for permission management across the platform. Furthermore, the integration of Stripe for payments is another design decision directly influenced by our stakeholder where they emphasized the need for secure payments. 
  
  \item While creating the design doc, what parts of your other documents (e.g.
  requirements, hazard analysis, etc), it any, needed to be changed, and why?

    While creating the design doc, multiple areas were identified where updates would be necessary. An example is in the hazard analysis such as potential authorization issues. These changes have been taken note of and will be updated in the next revision of the document.
  
  \item What are the limitations of your solution?  Put another way, given
  unlimited resources, what could you do to make the project better?(LO\_ProbSolutions)

    A limitation to our solution is our dependency on external services. For example, we are depending on Stripe to have 99.99\% uptime to meet our software needs. Another dependency is using the McMaster identity platform to sign-in and authenticate users. If we had the luxury of time and resources, we would opt to create our own payment gateway that met the security concerns of our stakeholder.
  
  \item Give a brief overview of other design solutions you considered.  What
  are the benefits and tradeoffs of those other designs compared with the chosen
  design?  From all the potential options, why did you select the documented design?
  (LO\_Explores)

    We considered many design solutions for our payment gateway such as paddle or square, but we settled with Stripe with our stakeholder for its simplicty and easy integration. We also considered many types of authentication for the platform such as creating our own authentication where users sign up and register with their own email. However, we decided to use the McMaster identity platfrom since it would be the easiest to verify users and keep users to only valid McMaster students.

  \end{enumerate}


\end{document}