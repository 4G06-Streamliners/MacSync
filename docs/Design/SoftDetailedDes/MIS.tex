\documentclass[12pt, titlepage]{article}

\usepackage{amsmath, mathtools}

\usepackage[round]{natbib}
\usepackage{amsfonts}
\usepackage{amssymb}
\usepackage{graphicx}
\usepackage{colortbl}
\usepackage{xr}
\usepackage{hyperref}
\usepackage{longtable}
\usepackage{xfrac}
\usepackage{tabularx}
\usepackage{float}
\usepackage{siunitx}
\usepackage{booktabs}
\usepackage{multirow}
\usepackage[section]{placeins}
\usepackage{caption}
\usepackage{fullpage}

\hypersetup{
bookmarks=true,     % show bookmarks bar?
colorlinks=true,       % false: boxed links; true: colored links
linkcolor=red,          % color of internal links (change box color with linkbordercolor)
citecolor=blue,      % color of links to bibliography
filecolor=magenta,  % color of file links
urlcolor=cyan          % color of external links
}

\usepackage{array}

\externaldocument{../../SRS/SRS}

\input{../../Comments}
%% Common Parts

\newcommand{\progname}{Software Engineering} % PUT YOUR PROGRAM NAME HERE
\newcommand{\authname}{Team \#12, Streamliners
\\ Mahad Ahmed
\\ Abyan Jaigirdar
\\ Prerna Prabhu
\\ Farhan Rahman
\\ Ali Zia} % AUTHOR NAMES                  

\usepackage{hyperref}
    \hypersetup{colorlinks=true, linkcolor=blue, citecolor=blue, filecolor=blue,
                urlcolor=blue, unicode=false}
    \urlstyle{same}     

\begin{document}

\title{Module Interface Specification for \progname{}}

\author{\authname}

\date{\today}

\maketitle

\pagenumbering{roman}

\section{Revision History}

\begin{tabularx}{\textwidth}{p{3cm}p{2cm}X}
\toprule {\bf Date} & {\bf Version} & {\bf Notes}\\
\midrule
Date 1 & 1.0 & Notes\\
Date 2 & 1.1 & Notes\\
\bottomrule
\end{tabularx}

~\newpage

\section{Symbols, Abbreviations and Acronyms}

See SRS Documentation at \wss{give url}

\wss{Also add any additional symbols, abbreviations or acronyms}

\newpage

\tableofcontents

\newpage

\pagenumbering{arabic}

\section{Introduction}

The following document details the Module Interface Specifications for
\wss{Fill in your project name and description}

Complementary documents include the System Requirement Specifications
and Module Guide.  The full documentation and implementation can be
found at \url{...}.  \wss{provide the url for your repo}

\section{Notation}

\wss{You should describe your notation.  You can use what is below as
  a starting point.}

The structure of the MIS for modules comes from \citet{HoffmanAndStrooper1995},
with the addition that template modules have been adapted from
\cite{GhezziEtAl2003}.  The mathematical notation comes from Chapter 3 of
\citet{HoffmanAndStrooper1995}.  For instance, the symbol := is used for a
multiple assignment statement and conditional rules follow the form $(c_1
\Rightarrow r_1 | c_2 \Rightarrow r_2 | ... | c_n \Rightarrow r_n )$.

The following table summarizes the primitive data types used by \progname. 

\begin{center}
\renewcommand{\arraystretch}{1.2}
\noindent 
\begin{tabular}{l l p{7.5cm}} 
\toprule 
\textbf{Data Type} & \textbf{Notation} & \textbf{Description}\\ 
\midrule
character & char & a single symbol or digit\\
integer & $\mathbb{Z}$ & a number without a fractional component in (-$\infty$, $\infty$) \\
natural number & $\mathbb{N}$ & a number without a fractional component in [1, $\infty$) \\
real & $\mathbb{R}$ & any number in (-$\infty$, $\infty$)\\
\bottomrule
\end{tabular} 
\end{center}

\noindent
The specification of \progname \ uses some derived data types: sequences, strings, and
tuples. Sequences are lists filled with elements of the same data type. Strings
are sequences of characters. Tuples contain a list of values, potentially of
different types. In addition, \progname \ uses functions, which
are defined by the data types of their inputs and outputs. Local functions are
described by giving their type signature followed by their specification.

\section{Module Decomposition}

The following table is taken directly from the Module Guide document for this project.

\begin{table}[h!]
\centering
\begin{tabular}{p{0.3\textwidth} p{0.6\textwidth}}
\toprule
\textbf{Level 1} & \textbf{Level 2}\\
\midrule

{Hardware-Hiding} & ~ \\
\midrule

\multirow{7}{0.3\textwidth}{Behaviour-Hiding} & Input Parameters\\
& Output Format\\
& Output Verification\\
& Temperature ODEs\\
& Energy Equations\\ 
& Control Module\\
& Specification Parameters Module\\
\midrule

\multirow{3}{0.3\textwidth}{Software Decision} & {Sequence Data Structure}\\
& ODE Solver\\
& Plotting\\
\bottomrule

\end{tabular}
\caption{Module Hierarchy}
\label{TblMH}
\end{table}

\newpage
~\newpage

\section{MIS of \wss{Module Name}} \label{Module} \wss{Use labels for
  cross-referencing}

\wss{You can reference SRS labels, such as R\ref{R_Inputs}.}

\wss{It is also possible to use \LaTeX for hypperlinks to external documents.}

\subsection{MIS of Registration Rules Module (M6)} \label{MIS_M6}

\subsubsection{Module}

Registration Rules Module (M6)

This module enforces all event-level registration constraints for 
Bus Sign-Up (M3.1), Table Sign-Up (M3.2), and RSVP Sign-Up (M3.3). It retrieves 
event rules such as capacity limits, table constraints, bus seating limits, 
registration deadlines, and RSVP policies from the backend datastore.  
It validates whether a user is allowed to register, whether an event has 
remaining capacity, and whether any additional restrictions apply.  
All rule enforcement occurs before entries are recorded or notifications are triggered.

\subsubsection{Uses}

\begin{itemize}
    \item Access Control Module (M5): for checking user roles and eligibility
    \item User Authorization Module (M8): for identity data required in rule checks
    \item Backend datastore (via data access layer): for retrieving event rules and counts
    \item Sign-Up Module (M3): requests rule validation during sign-up flows
\end{itemize}

\subsection{Syntax}

\subsubsection{Exported Constants}

N/A

\subsubsection{Exported Access Programs}

\begin{center}
\renewcommand{\arraystretch}{1.2}
\begin{tabularx}{\textwidth}{l l l X}
\toprule
\textbf{Name} & \textbf{In} & \textbf{Out} & \textbf{Exceptions} \\
\midrule
validateCapacity     & EventInfo, UserInfo & Boolean & EventFull, InvalidEvent \\
validateDeadline     & EventInfo, UserInfo & Boolean & DeadlinePassed, InvalidEvent \\
validateEligibility  & EventInfo, UserInfo & Boolean & PermissionDenied \\
getEventRules        & EventID             & EventRules & InvalidEvent \\
\bottomrule
\end{tabularx}
\end{center}

\subsection{Semantics}

\subsubsection{State Variables}

None.

\subsubsection{Environment Variables}

\begin{itemize}
    \item Database connection (event metadata, capacity counts, deadlines)
\end{itemize}

\subsubsection{Assumptions}

\begin{itemize}
    \item The database contains valid and up-to-date event policies.
    \item EventInfo passed in from M3 contains a valid event ID.
    \item UserInfo contains a valid authenticated user ID.
\end{itemize}

\subsubsection{Access Routine Semantics}

\noindent \textbf{validateCapacity}(eventInfo, userInfo):

\begin{itemize}
    \item transition: N/A
    \item output: Returns true if event capacity has not been reached.
    \item exception: EventFull if capacity is exhausted, InvalidEvent if event does not exist.
\end{itemize}

\vspace{0.15cm}

\noindent \textbf{validateDeadline}(eventInfo, userInfo):

\begin{itemize}
    \item transition: N/A
    \item output: Returns true if registration is still open.
    \item exception: DeadlinePassed if the cutoff time has passed, InvalidEvent otherwise.
\end{itemize}

\vspace{0.15cm}

\noindent \textbf{validateEligibility}(eventInfo, userInfo):

\begin{itemize}
    \item transition: N/A
    \item output: Returns true if the user meets role-based or event-specific criteria.
    \item exception: PermissionDenied if the user is not allowed to register.
\end{itemize}

\vspace{0.15cm}

\noindent \textbf{getEventRules}(eventID):

\begin{itemize}
    \item transition: N/A
    \item output: Returns the complete rule set for the given event (capacity, deadlines, policies).
    \item exception: InvalidEvent if the event does not exist.
\end{itemize}

\subsubsection{Local Functions}

\begin{itemize}
    \item lookupRules(eventID): fetches the rule set for an event from the datastore.
    \item isBeforeDeadline(eventInfo): compares current time to stored deadline.
    \item hasRemainingCapacity(eventInfo): queries current participant count.
    \item meetsEligibilityCriteria(eventInfo, userInfo): checks role access and custom restrictions.
\end{itemize}

\newpage

\bibliographystyle {plainnat}
\bibliography {../../../refs/References}

\newpage

\section{Appendix} \label{Appendix}

\wss{Extra information if required}

\newpage{}

\section*{Appendix --- Reflection}

\wss{Not required for CAS 741 projects}

The information in this section will be used to evaluate the team members on the
graduate attribute of Problem Analysis and Design.

\input{../../Reflection.tex}

\begin{enumerate}
  \item What went well while writing this deliverable? 
  \item What pain points did you experience during this deliverable, and how
    did you resolve them?
  \item Which of your design decisions stemmed from speaking to your client(s)
  or a proxy (e.g. your peers, stakeholders, potential users)? For those that
  were not, why, and where did they come from?
  \item While creating the design doc, what parts of your other documents (e.g.
  requirements, hazard analysis, etc), it any, needed to be changed, and why?
  \item What are the limitations of your solution?  Put another way, given
  unlimited resources, what could you do to make the project better? (LO\_ProbSolutions)
  \item Give a brief overview of other design solutions you considered.  What
  are the benefits and tradeoffs of those other designs compared with the chosen
  design?  From all the potential options, why did you select the documented design?
  (LO\_Explores)
\end{enumerate}


\end{document}