\documentclass[12pt, titlepage]{article}

\usepackage{fullpage}
\usepackage[round]{natbib}
\usepackage{multirow}
\usepackage{booktabs}
\usepackage{tabularx}
\usepackage{graphicx}
\usepackage{float}
\usepackage{hyperref}
\hypersetup{
    colorlinks,
    citecolor=blue,
    filecolor=black,
    linkcolor=red,
    urlcolor=blue
}

\input{../../Comments}
%% Common Parts

\newcommand{\progname}{Software Engineering} % PUT YOUR PROGRAM NAME HERE
\newcommand{\authname}{Team \#12, Streamliners
\\ Mahad Ahmed
\\ Abyan Jaigirdar
\\ Prerna Prabhu
\\ Farhan Rahman
\\ Ali Zia} % AUTHOR NAMES                  

\usepackage{hyperref}
    \hypersetup{colorlinks=true, linkcolor=blue, citecolor=blue, filecolor=blue,
                urlcolor=blue, unicode=false}
    \urlstyle{same}     

\newcounter{acnum}
\newcommand{\actheacnum}{AC\theacnum}
\newcommand{\acref}[1]{AC\ref{#1}}

\newcounter{ucnum}
\newcommand{\uctheucnum}{UC\theucnum}
\newcommand{\uref}[1]{UC\ref{#1}}

\newcounter{mnum}
\newcommand{\mthemnum}{M\themnum}
\newcommand{\mref}[1]{M\ref{#1}}
\newcounter{submnum}

\begin{document}

\title{Module Guide for \progname{}} 
\author{\authname}
\date{\today}

\maketitle

\pagenumbering{roman}

\section{Revision History}

\begin{tabularx}{\textwidth}{p{3cm}p{2cm}X}
\toprule {\bf Date} & {\bf Version} & {\bf Notes}\\
\midrule
Date 1 & 1.0 & Notes\\
Date 2 & 1.1 & Notes\\
\bottomrule
\end{tabularx}

\newpage

\section{Reference Material}

This section records information for easy reference.

\subsection{Abbreviations and Acronyms}

\renewcommand{\arraystretch}{1.2}
\begin{tabular}{l l} 
  \toprule		
  \textbf{symbol} & \textbf{description}\\
  \midrule 
  AC & Anticipated Change\\
  DAG & Directed Acyclic Graph \\
  M & Module \\
  MG & Module Guide \\
  OS & Operating System \\
  R & Requirement\\
  SC & Scientific Computing \\
  SRS & Software Requirements Specification\\
  \progname & Explanation of program name\\
  UC & Unlikely Change \\
  \wss{etc.} & \wss{...}\\
  \bottomrule
\end{tabular}\\

\newpage

\tableofcontents

\listoftables

\listoffigures

\newpage

\pagenumbering{arabic}

\section{Introduction}

Decomposing a system into modules is a commonly accepted approach to developing
software.  A module is a work assignment for a programmer or programming
team~\citep{ParnasEtAl1984}.  We advocate a decomposition
based on the principle of information hiding~\citep{Parnas1972a}.  This
principle supports design for change, because the ``secrets'' that each module
hides represent likely future changes.  Design for change is valuable in SC,
where modifications are frequent, especially during initial development as the
solution space is explored.  

Our design follows the rules layed out by \citet{ParnasEtAl1984}, as follows:
\begin{itemize}
\item System details that are likely to change independently should be the
  secrets of separate modules.
\item Each data structure is implemented in only one module.
\item Any other program that requires information stored in a module's data
  structures must obtain it by calling access programs belonging to that module.
\end{itemize}

After completing the first stage of the design, the Software Requirements
Specification (SRS), the Module Guide (MG) is developed~\citep{ParnasEtAl1984}. The MG
specifies the modular structure of the system and is intended to allow both
designers and maintainers to easily identify the parts of the software.  The
potential readers of this document are as follows:

\begin{itemize}
\item New project members: This document can be a guide for a new project member
  to easily understand the overall structure and quickly find the
  relevant modules they are searching for.
\item Maintainers: The hierarchical structure of the module guide improves the
  maintainers' understanding when they need to make changes to the system. It is
  important for a maintainer to update the relevant sections of the document
  after changes have been made.
\item Designers: Once the module guide has been written, it can be used to
  check for consistency, feasibility, and flexibility. Designers can verify the
  system in various ways, such as consistency among modules, feasibility of the
  decomposition, and flexibility of the design.
\end{itemize}

The rest of the document is organized as follows. Section
\ref{SecChange} lists the anticipated and unlikely changes of the software
requirements. Section \ref{SecMH} summarizes the module decomposition that
was constructed according to the likely changes. Section \ref{SecConnection}
specifies the connections between the software requirements and the
modules. Section \ref{SecMD} gives a detailed description of the
modules. Section \ref{SecTM} includes two traceability matrices. One checks
the completeness of the design against the requirements provided in the SRS. The
other shows the relation between anticipated changes and the modules. Section
\ref{SecUse} describes the use relation between modules.

\section{Anticipated and Unlikely Changes} \label{SecChange}

This section lists possible changes to the system. According to the likeliness
of the change, the possible changes are classified into two
categories. Anticipated changes are listed in Section \ref{SecAchange}, and
unlikely changes are listed in Section \ref{SecUchange}.

\subsection{Anticipated Changes} \label{SecAchange}

Anticipated changes are the source of the information that is to be hidden
inside the modules. Ideally, changing one of the anticipated changes will only
require changing the one module that hides the associated decision. The approach
adapted here is called design for
change.

\makeatletter
\renewcommand{\p@mnum}{M}
\makeatother

\begin{description}
  
\item[\refstepcounter{acnum} \actheacnum \label{acPaymentGateway}:] 
Future updates to the \textbf{Payment Processing Module}~(\ref{M1}) may require migration 
to different payment providers (e.g., Stripe, PayPal, Square) or McMaster’s internal 
e-commerce API.

\item[\refstepcounter{acnum} \actheacnum \label{acSignupTypes}:] 
New event types (e.g., workshops, volunteer shifts) may require additional registration 
flows beyond bus, table, and RSVP sign-ups. 
\textbf{Affected Module:} Sign-Up Module~(\ref{M3}).

\item[\refstepcounter{acnum} \actheacnum \label{acNotifications}:] 
Notification delivery methods or content templates may expand to include SMS or 
in-app alerts in addition to email and push notifications. 
\textbf{Affected Module:} Notification Handling Module (\ref{M7}).

\item[\refstepcounter{acnum} \actheacnum \label{acNotifications}:] 
Notification delivery methods or content templates may expand to include SMS or 
in-app alerts in addition to email and push notifications. 
\textbf{Affected Module:} Notification Handling Module (\ref{M7}).

\item[\refstepcounter{acnum} \actheacnum \label{acPolicyRules}:] 
Event policies such as capacity limits, sign-up deadlines, or priority access 
rules may change based on MES regulations. 
\textbf{Affected Module:} Registration Rules (\ref{M6}).

\item[\refstepcounter{acnum} \actheacnum \label{acAnalytics}:] 
Admin dashboard metrics and analytics may expand with new reporting filters 
or visualizations. 
\textbf{Affected Module:} Admin Dashboard and Analytics (\ref{M5}).

\end{description}

\subsection{Unlikely Changes} \label{SecUchange}

The module design should be as general as possible. However, a general system is
more complex. Sometimes this complexity is not necessary. Fixing some design
decisions at the system architecture stage can simplify the software design. If
these decision should later need to be changed, then many parts of the design
will potentially need to be modified. Hence, it is not intended that these
decisions will be changed.

\begin{description}

\item[\refstepcounter{ucnum} \uctheucnum \label{ucTechStack}:] 
The system's technology stack (Next.js/React frontend, NestJS backend, PostgreSQL database) 
is a stable decision and unlikely to change during or after development.
Altering this foundation would require complete re-architecture.

\item[\refstepcounter{ucnum} \uctheucnum \label{ucHosting}:] 
Hosting on DigitalOcean and database management using PostgreSQL containers 
are fixed infrastructure decisions.

\item[\refstepcounter{ucnum} \uctheucnum \label{ucPrivacy}:] 
Compliance with PIPEDA and McMaster University privacy policies is a permanent 
requirement that will not change. Any change would necessitate legal
review and cross-module rewriting.

\item[\refstepcounter{ucnum} \uctheucnum \label{ucRolesArch}:] 
Security protocols such as TLS 1.3, AES-256 encryption, audit logging, and 
fail-safe defaults are core safety standards defined in the Hazard Analysis
and are not expected to change.

\item[\refstepcounter{ucnum} \uctheucnum \label{ucRolesArch}:] 
The role hierarchy (Attendee, Organizer, Executive) and three-tier architecture 
(frontend, backend, database) are foundational and not expected to evolve.

\item[\refstepcounter{ucnum} \uctheucnum \label{ucScope}:] 
The system scope will remain focused on MES event management. Expansion to unrelated 
domains (e.g., general club management or social networking) is not anticipated.

\end{description}

\section{Module Hierarchy} \label{SecMH}

This section provides an overview of the module design. Modules are summarized
in a hierarchy decomposed by secrets in Table \ref{TblMH}. The modules listed
below, which are leaves in the hierarchy tree, are the modules that will
actually be implemented.

\subsection*{User Interface Modules}

\begin{description}
\refstepcounter{mnum}\label{M1}
\item [M\themnum: Payment Processing Module] Handles the logic and communication between
the frontend and backend services for processing event payments and verifying
transactions.

\refstepcounter{mnum}\label{M2}
\item [M\themnum: RBAC/FBAC Access Control Module] Manages user permissions and determines
which features are accessible to each user role within the mobile and web
applications. This module relies on the User Authorization Module for user
authentication and identity verification.

\refstepcounter{mnum}\label{M3}
\item [M\themnum: Sign-Up Module] Provides functionality for participants to register for
different event activities. It contains shared logic for registration flows,
along with specific submodules for each type of sign-up.
\renewcommand{\thesubmnum}{\themnum.\arabic{submnum}}
\begin{itemize}
    \item [\refstepcounter{submnum}\label{M3.1}\textbf{M\thesubmnum:}] \textbf{Bus Sign-Up Module} Handles registration for bus
    transportation, including capacity validation and seat allocation.
    \item [\refstepcounter{submnum}\label{M3.2}\textbf{M\thesubmnum:}] \textbf{Table Sign-Up Module} Manages registration and table
    assignments for group events.
    \item [\refstepcounter{submnum}\label{M3.3}\textbf{M\thesubmnum:}] \textbf{RSVP Sign-Up Module} Handles RSVP submissions and
    confirmation tracking for events.
\end{itemize}
\end{description}

\subsection*{Backend Modules}

\begin{description}
\refstepcounter{mnum}\label{M4}
\item [M\themnum: Payment Configuration Module] Stores and manages payment parameters such
as currency, and API integration details for supported payment
methods.

\refstepcounter{mnum}\label{M5}
\item [M\themnum: Access Control Module] Implements backend logic for role-based and
feature-based permissions, ensuring consistent authorization across the system.

\refstepcounter{mnum}\label{M6}
\item [M\themnum: Registration Rules Module] Defines and enforces sign-up policies,
including event deadlines, capacity limits, and participant prioritization
criteria.

\refstepcounter{mnum}\label{M7}
\item [M\themnum: Notification Handling Module] Manages the delivery of notifications to
users across different parts of the system. It coordinates sending confirmations,
updates, and reminders related to payments and registrations. This module ensures
that users receive timely and relevant information through email or in-app alerts.

\refstepcounter{mnum}\label{M8}
\item [M\themnum: User Authorization Module] Handles user authentication, login, and
session management. It verifies user credentials, and
provides identity data to other modules such as RBAC/FBAC and Payment Processing.
This module serves as the foundation for all user-based operations within the system.
\end{description}

\begin{table}[H]
\centering
\begin{tabular}{p{0.3\textwidth} p{0.6\textwidth}}
\toprule
\textbf{Level 1} & \textbf{Level 2}\\
\midrule

{Hardware-Hiding Module} & ~ \\
\midrule

\multirow{8}{0.3\textwidth}{Behaviour-Hiding Module} 
& Payment Processing Module\\
& RBAC/FBAC Access Control Module\\
& Sign-Up Module\\
& \hspace{1em} Bus Sign-Up Module\\
& \hspace{1em} Table Sign-Up Module\\
& \hspace{1em} RSVP Sign-Up Module\\
& User Authorization Module\\
\midrule

\multirow{4}{0.3\textwidth}{Software Decision Module}
& Payment Configuration Module\\
& Access Control Module\\
& Registration Rules Module\\
& Notification Handling Module\\
\bottomrule
\end{tabular}
\caption{Module Hierarchy}
\label{TblMH}
\end{table}


\section{Connection Between Requirements and Design} \label{SecConnection}

The design of the Large Event Management System is intended to satisfy the functional
requirements (F101–F116) and the associated non-functional requirements identified in the
Software Requirements Specification (SRS). This section explains how these requirements
guided the modular decomposition of the system into the components presented in
Section~\ref{SecMH}. The connection between requirements and modules is listed in Table~\ref{TblRT}.

Requirements related to \textbf{event registration and ticketing} (F101–F104) are
satisfied through the \textit{Sign-Up Module (M3)} and its submodules
(\textit{M3.1–M3.4}). These modules handle all user registration flows including ticket
purchasing, bus and table sign-ups, RSVPs, and refund management. They interact with the
\textit{Registration Rules Module (M6)} to enforce deadlines, seat capacities, and
prioritization policies, ensuring that registration logic remains consistent and adaptable
to different event configurations.

\textbf{Payment-related requirements} (F105–F107) are fulfilled by the
\textit{Payment Processing Module (M1)} and the \textit{Payment Configuration Module
(M4)}. Together they manage secure payment gateway integration, transaction verification,
and storage of configuration data such as currency types, merchant keys, and payment
methods. This modular separation supports both extensibility—allowing future integration
with new payment providers—and maintainability by isolating external API dependencies.

\textbf{Sign-up and allocation requirements} (F108–F110), including bus registration,
table assignments, and RSVP limits, are handled by specific submodules within the
\textit{Sign-Up Module (M3)}. Each submodule provides a tailored workflow for its
respective function, while common logic is centralized through the \textit{Sign-Up
Adapter (M3.1)}. This satisfies the SRS need for flexible and reusable registration
components that can be applied across multiple event types.

\textbf{Access-control requirements} (F111–F113) motivated the inclusion of both the
\textit{RBAC/FBAC Access Control Module (M2)} and the backend
\textit{Access Control Module (M5)}. These modules collectively enforce role-based and
feature-based access restrictions, ensuring that users—such as administrators, volunteers,
or attendees—only interact with features appropriate to their permissions. This design
choice provides fine-grained authorization while remaining scalable for future event roles.

\textbf{Security, privacy, and integrity requirements} (F114–F116) are addressed through
the \textit{User Authorization Module (M8)} in conjunction with other backend modules
(\textit{M4–M7}). The authorization module manages user authentication and secure session
handling, while supporting encryption and data-protection mechanisms across network
communications. These features guarantee that sensitive user and payment data comply with
privacy regulations and institutional security standards.

Finally, cross-cutting requirements for \textbf{notifications, confirmations, and user
communication} are met by the \textit{Notification Handling Module (M7)}. This module
delivers confirmations, reminders, and updates via email and in-app alerts, ensuring that
both attendees and administrators receive timely feedback on their actions within the
system.

By associating each major requirement group in the SRS with a dedicated set of modules,
the overall architecture remains flexible, maintainable, and extensible. This structured
approach ensures that any future modifications—such as supporting new payment methods,
adjusting registration logic, or introducing new user roles—can be achieved by updating a
limited subset of modules. The detailed requirement–module correspondence is provided in
Table~\ref{TblRT} in Section~\ref{SecTM}.

\section{Module Decomposition} \label{SecMD}

Modules are decomposed according to the principle of ``information hiding''
proposed by \citet{ParnasEtAl1984}. The \emph{Secrets} field in a module
decomposition is a brief statement of the design decision hidden by the
module. The \emph{Services} field specifies \emph{what} the module will do
without documenting \emph{how} to do it. For each module, a suggestion for the
implementing software is given under the \emph{Implemented By} title. If the
entry is \emph{OS}, this means that the module is provided by the operating
system or by standard programming language libraries.  \emph{\progname{}} means the
module will be implemented by the \progname{} software.

Only the leaf modules in the hierarchy have to be implemented. If a dash
(\emph{--}) is shown, this means that the module is not a leaf and will not have
to be implemented.

\subsection{Hardware Hiding Modules (\mref{mHH})}

\begin{description}
\item[Secrets:]The data structure and algorithm used to implement the virtual
  hardware.
\item[Services:]Serves as a virtual hardware used by the rest of the
  system. This module provides the interface between the hardware and the
  software. So, the system can use it to display outputs or to accept inputs.
\item[Implemented By:] OS
\end{description}

\subsection{Behaviour-Hiding Module}

\begin{description}
\item[Secrets:]The contents of the required behaviours.
\item[Services:]Includes programs that provide externally visible behaviour of
  the system as specified in the software requirements specification (SRS)
  documents. This module serves as a communication layer between the
  hardware-hiding module and the software decision module. The programs in this
  module will need to change if there are changes in the SRS.
\item[Implemented By:] --
\end{description}

\subsubsection{Payment Processing Module (\mref{M1})}

\begin{description}
\item[Secrets:] Payment gateway integration, logic for the transaction workflows.
\item[Services:] Allows for client to fulfill payment requests, executes charges/refunds, and reports all transactions to logs.
\item[Implemented By:] \progname{}
\item[Type of Module:] Abstract Object
\end{description}

\subsubsection{RBAC/FBAC Access Control Module (\mref{M2})}

\begin{description}
\item[Secrets:] The mapping of roles to features that allow capabilities in the UI.
\item[Services:] Determines what views/actions are allowed for a signed in user. 
\item[Implemented By:] \progname{}
\item[Type of Module:] Library Component
\end{description}

\subsubsection{Sign-Up Module (\mref{M3})}

\begin{description}
\item[Secrets:] Common sign up state machine (registered, confirmed, waitlisted).
\item[Services:] Shared logic for creating and managing signups, delegates concrete services to submodules.  
\item[Implemented By:] \progname{}
\item[Type of Module:] Abstract Object
\end{description}

\subsubsection{Bus Sign-up Module (\mref{M3.1})}

\begin{description}
\item[Secrets:] Capacity model, seat allocation logic and, waitlisting logic.
\item[Services:] Allows users to reserve bus seat while enforcing capacity.
\item[Implemented By:] \progname{}
\item[Type of Module:] Abstract Object
\end{description}

\subsubsection{Table Sign-up Module (\mref{M3.2})}

\begin{description}
\item[Secrets:] Table size, number of tables, and grouping constraints.
\item[Services:] Allows event organizers to create/manage table groups.
\item[Implemented By:] \progname{}
\item[Type of Module:] Abstract Object
\end{description}

\subsubsection{RSVP Sign-up Module (\mref{M3.3})}

\begin{description}
\item[Secrets:] Event capacity, RSVP deadlines. 
\item[Services:] Allows users to RSVP for events. enforces constraints/limits and reports user status.
\item[Implemented By:] \progname{}
\item[Type of Module:] Abstract Object
\end{description}


\subsubsection{User Authorization Module (\mref{M8})}

\begin{description}
\item[Secrets:] Identity verification, session tokens. 
\item[Services:] Authenticates the user, issues and validates session tokens, provides identity context to \mref{M2}.
\item[Implemented By:] \progname{}
\item[Type of Module:] Library
\end{description}

\subsection{Software Decision Module}

\begin{description}
\item[Secrets:] The design decision based on mathematical theorems, physical
  facts, or programming considerations. The secrets of this module are
  \emph{not} described in the SRS.
\item[Services:] Includes data structure and algorithms used in the system that
  do not provide direct interaction with the user. 
  % Changes in these modules are more likely to be motivated by a desire to
  % improve performance than by externally imposed changes.
\item[Implemented By:] --
\end{description}

\subsubsection{Payment Configuration Module (\mref{M4})}

\begin{description}
\item[Secrets:] Supported providers, external/internal services, pricing schemas.
\item[Services:] Stores and provides payment parameters to \mref{M1}. 
\item[Implemented By:] \progname{}
\item[Type of Module:] Library Component
\end{description}

\subsubsection{Access Control Module (\mref{M5})}

\begin{description}
\item[Secrets:] Logic to evaluate permissions and stores all policies.
\item[Services:] Evaluates actions being performed by the user and grants them the correct views/resources. 
\item[Implemented By:] \progname{}
\item[Type of Module:] Library
\end{description}


\subsubsection{Registration Rules Module (\mref{M6})}

\begin{description}
\item[Secrets:] Global event constraints (capacity, deadlines) holds all the default event details/policies.
\item[Services:] Validates sign up operations across all sign up modules. Provides \mref{M3} with core information about events.
\item[Implemented By:] \progname{}
\item[Type of Module:] Abstract Object
\end{description}

\subsubsection{Notification Handling Module (\mref{M7})}

\begin{description}
\item[Secrets:] Integrations (emails/push), notification templating, retry policies.
\item[Services:] Sends confirmations, reminders and status updates triggered by \mref{M1}/\mref{M3}.
\item[Implemented By:] \progname{}
\item[Type of Module:] Library Component.
\end{description}

\section{Traceability Matrix} \label{SecTM}

This section presents two traceability matrices: the first shows how each
functional requirement from the SRS (F101–F116) is satisfied by one or more
modules; the second links the anticipated changes identified in
Section~\ref{SecChange} to the modules that would be affected.

\begin{table}[H]
\centering
\begin{tabular}{p{0.22\textwidth} p{0.68\textwidth}}
\toprule
\textbf{Req. ID} & \textbf{Modules Implementing Requirement}\\
\midrule
F101--F104 & \mref{M3} (Sign-Up Module), \mref{M3.1--M3.4} (Sign-Up Submodules), \mref{M6} (Registration Rules Module)\\
F105--F107 & \mref{M1} (Payment Processing Module), \mref{M4} (Payment Configuration Module)\\
F108--F110 & \mref{M3} (Sign-Up Module), \mref{M3.1--M3.4} (Sign-Up Submodules), \mref{M6} (Registration Rules Module)\\
F111--F113 & \mref{M2} (RBAC/FBAC Access Control Module), \mref{M5} (Access Control Module)\\
F114--F116 & \mref{M8} (User Authorization Module), \mref{M4--M7} (Backend Modules)\\
\bottomrule
\end{tabular}
\caption{Trace Between Requirements and Modules}
\label{TblRT}
\end{table}

\begin{table}[H]
\centering
\begin{tabular}{p{0.28\textwidth} p{0.62\textwidth}}
\toprule
\textbf{Anticipated Change (AC)} & \textbf{Modules Affected}\\
\midrule
\acref{acPaymentGateway} & \mref{M1} (Payment Processing Module), \mref{M4} (Payment Configuration Module)\\
\acref{acSignupTypes} & \mref{M3} (Sign-Up Module), \mref{M6} (Registration Rules Module)\\
\acref{acNotifications} & \mref{M7} (Notification Handling Module)\\
\acref{acPolicyRules} & \mref{M6} (Registration Rules Module), \mref{M3} (Sign-Up Submodules)\\
\acref{acAnalytics} & \mref{M5} (Access Control Module), \mref{M7} (Notification Handling Module)\\
\bottomrule
\end{tabular}
\caption{Trace Between Anticipated Changes and Modules}
\label{TblACT}
\end{table}

\section{Use Hierarchy Between Modules} \label{SecUse}

In this section, the uses hierarchy between modules is
provided. \citet{Parnas1978} said of two programs A and B that A {\em uses} B if
correct execution of B may be necessary for A to complete the task described in
its specification. That is, A {\em uses} B if there exist situations in which
the correct functioning of A depends upon the availability of a correct
implementation of B.  Figure \ref{FigUH} illustrates the use relation between
the modules. It can be seen that the graph is a directed acyclic graph
(DAG). Each level of the hierarchy offers a testable and usable subset of the
system, and modules in the higher level of the hierarchy are essentially simpler
because they use modules from the lower levels.

\begin{figure}[H]
\centering
\includegraphics[width=0.9\textwidth]{UsesHierarchy.png}
\caption{Use hierarchy among modules}
\label{FigUH}
\end{figure}

%\section*{References}

\section{User Interfaces}

\begin{figure}[H]
    \centering
    \includegraphics[width=0.8\textwidth]{Signup.png}
    \caption{Signup Page}
\end{figure}

\begin{figure}[H]
    \centering
    \includegraphics[width=0.8\textwidth]{Payment.png}
    \caption{Payment Page}
\end{figure}

\begin{figure}[H]
    \centering
    \includegraphics[width=0.8\textwidth]{Access_Control.png}
    \caption{Role-based/Feautre-based Access Control}
\end{figure}

\section{Design of Communication Protocols}

This section outlines the external services that will be used by the MacSync platform ensuring appropriate integration between the application and third-party systems.

\subsection{External Services Overview}

The MacSync platform will interface with 2 key external APIs to support event management functions:
\begin{itemize}
    \item \textbf{Stripe API} for secure online payments.
    \item \textbf{McMaster MacID SSO} to authenticate and authorise users
\end{itemize}

\subsection{Payment Service: Stripe}

Stripe will be used as the primary gateway for handling all necessary transcations for registering for events. All communication with Stripe occurs over secure HTTPS via Stripes RESTful API. All transaction records will be recorded, logged and stored within our database to maintain past records. This enables a secure way to integrate payments within our application without having to worry about the implications of building a custom payment stack.

\subsection{Authentication and Authorization: McMaster MacID SSO}

The MacSync platform will integrate with McMaster's identity platform to authenticate users and support \mref{M2}. After logging in via MacID their profile will be linked to key information such as unique identifier (MacID), email address, given name, and affiliation (student, employee). By using the university SSO, we ensure that only valid McMaster students are registered on the platform and reduces authentication overhead.


\section{Timeline}

\wss{Schedule of tasks and who is responsible}

\wss{You can point to GitHub if this information is included there}

\bibliographystyle {plainnat}
\bibliography{../../../refs/References}

\newpage{}

\end{document}
