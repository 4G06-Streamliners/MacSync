\documentclass{article}

\usepackage{float}
\restylefloat{table}

\usepackage{booktabs}

\title{Team Contributions: Rev 0\\\progname}

\author{\authname}

\date{}

\input{../Comments}
%% Common Parts

\newcommand{\progname}{Software Engineering} % PUT YOUR PROGRAM NAME HERE
\newcommand{\authname}{Team \#12, Streamliners
\\ Mahad Ahmed
\\ Abyan Jaigirdar
\\ Prerna Prabhu
\\ Farhan Rahman
\\ Ali Zia} % AUTHOR NAMES                  

\usepackage{hyperref}
    \hypersetup{colorlinks=true, linkcolor=blue, citecolor=blue, filecolor=blue,
                urlcolor=blue, unicode=false}
    \urlstyle{same}     

\begin{document}

\maketitle

This document summarizes the contributions of each team member for the Rev 0
Demo.  The time period of interest is the time between the PoC demo and the Rev
0 demo; the contributions prior to the PoC are NOT included.

\section{Demo Plans}

Our Rev 0 demo will demonstrate the complete system workflow and integration of all major components. The demo will begin with user sign-up and login using MAC address authentication, followed by event selection and registration. The flow will then incorporate event table/bus sign-ups with seat selection, demonstrating how users register for events and reserve available seats, followed by transaction processing through the payment system. The demo will also highlight role-based access control by presenting different user and administrator views, including the administrator dashboard for managing events, viewing registrations, and monitoring system activity. This end-to-end demonstration will show how the system supports event registration, payments, and management within a unified platform.

\section{Team Meeting Attendance}

\begin{table}[H]
\centering
\begin{tabular}{ll}
\toprule
\textbf{Student} & \textbf{Meetings}\\
\midrule
Total & 8\\
Mahad Ahmed & 8\\
Abyan Jaigirdar & 8\\
Prerna Prabhu & 8\\
Farhan Rahman & 8\\
Ali Zia & 8\\
\bottomrule
\end{tabular}
\end{table}

\noindent
The team arranges meetings almost weekly, to clarify objectives and align on priorities on major deliverables, with additional meetings scheduled as needed, typically in the days leading up to major deadlines. We have held a total of eight meetings in the time period of interest. Every team member has been able to attend these meetings, even if occasionally arriving slightly late or needing to leave early.

\section{Supervisor/Stakeholder Meeting Attendance}

\noindent \textbf{Supervisor's Name: } Luke Schuurman

\begin{table}[H]
\centering
\begin{tabular}{ll}
\toprule
\textbf{Student} & \textbf{Meetings}\\
\midrule
Total & 2\\
Abyan & 2\\
Ali & 2\\
Farhan & 2\\
Mahad & 2\\
Prerna & 2\\
\bottomrule
\end{tabular}
\end{table}

\noindent
Supervisor meetings were arranged when objectives and priorities needed to be clarified and all members attended those meetings. 

\section{Lecture Attendance}

Between the POC demo and the Rev 0 demo, there has been approximately 1 lecture held as part of the course schedule.

\begin{table}[H]
\centering
\begin{tabular}{ll}
\toprule
\textbf{Student} & \textbf{Lectures}\\
\midrule
Total & 1\\
Mahad Ahmed & 0\\
Abyan Jaigirdar & 0\\
Prerna Prabhu & 0\\
Farhan Rahman & 0\\
Ali Zia & 0\\
\bottomrule
\end{tabular}
\end{table}

\noindent The team couldn't make the scheduled lecture because we believed the unpredictable weather made the commute unsafe and unreasonable
for us that day, especially since majority of the team has 1.5+ hour commute one-way. Prerna, who lives closer to campus, could not
attend either due to a scheduling conflict.


\section{TA Document Discussion Attendance}


\noindent \textbf{TA's Name: } Tiago de Moraes Machado

\begin{table}[H]
\centering
\begin{tabular}{ll}
\toprule
\textbf{Student} & \textbf{Lectures}\\
\midrule
Total & 2\\
Ali Zia & 1\\
Abyan Jaigirdar & 2\\
Mahad Ahmed & 2\\
Farhan Rahman & 2\\
Prerna Prabhu & 2\\
\bottomrule
\end{tabular}
\end{table}

\noindent During the time period between the PoC and Rev 0 document deadlines, there were two TA document discussion sessions, including the POC demo.
Originally there were three sessions scheduled, but one was cancelled due to the university closure due to inclement weather. Ali was unable to attend
one of the sessions as he was sick that day.

\section{Commits}

\wss{For each team member how many commits to the main branch have been made
over the time period of interest.  The total is the total number of commits for
the entire team since the beginning of the term.  The percentage is the
percentage of the total commits made by each team member.}

\begin{table}[H]
\centering
\begin{tabular}{lll}
\toprule
\textbf{Student} & \textbf{Commits} & \textbf{Percent}\\
\midrule
Total & Num & 100\% \\
Name 1 & Num & \% \\
Name 2 & Num & \% \\
Name 3 & Num & \% \\
Name 4 & Num & \% \\
Name 5 & Num & \% \\
\bottomrule
\end{tabular}
\end{table}

\wss{If needed, an explanation for the counts can be provided here.  For
instance, if a team member has more commits to unmerged branches, these numbers
can be provided here.  If multiple people contribute to a commit, git allows for
multi-author commits.}

\section{Issue Tracker}

\wss{For each team member how many issues have they authored (including open and
closed issues (O+C)) and how many have they been assigned (only counting closed
issues (C only)) over the time period of interest.}

\begin{table}[H]
\centering
\begin{tabular}{lll}
\toprule
\textbf{Student} & \textbf{Authored (O+C)} & \textbf{Assigned (C only)}\\
\midrule
Name 1 & Num & Num \\
Name 2 & Num & Num \\
Name 3 & Num & Num \\
Name 4 & Num & Num \\
Name 5 & Num & Num \\
\bottomrule
\end{tabular}
\end{table}

\wss{If needed, an explanation for the counts can be provided here.}

\section{CICD}

Continuous Integration and Continuous Deployment (CI/CD) will be used to keep the project stable and easy to maintain as new features are added. The team will use GitHub Actions to automatically build, and test the system.

For \textbf{Continuous Integration (CI)}, every time a pull request is opened, the pipeline will automatically run linting, type checks, and unit tests. This helps catch issues early and ensures that the codebase stays clean and consistent. It will also build the project to make sure that new changes do not break existing functionality before they are merged.

For \textbf{Continuous Deployment (CD)}, after merging into the \texttt{dev} branch, a staging build will be automatically deployed for testing and feedback. When changes are merged into the \texttt{main} branch, a production build will be deployed. This keeps updates smooth and reduces the chance of deployment errors.

GitHub Actions will also send build or test failure notifications to the team so that problems can be fixed quickly. Environment variables and API keys will be stored securely using GitHub’s built-in secret management. Over time, the team may expand the pipeline to include integration or end-to-end tests, but for now the main goal is to automate building and testing to save time and improve reliability.

\section{Team Charter Trigger Items}

\subsection{Summary of Triggers}
The team charter established several quantified triggers to help maintain accountability and ensure consistent contribution. These include:
\begin{itemize}
    \item \textbf{Attendance:} Members are expected to maintain a 100\% attendance rate for all scheduled meetings unless an acceptable excuse (such as a health issue or family emergency) is provided.
    \item \textbf{GitHub Activity:} Each member must demonstrate consistent GitHub activity every week, with issues actively in progress or commits made for review.
    \item \textbf{Task Completion:} All assigned work must be completed and delivered on time according to deadlines set during weekly meetings.
\end{itemize}

\subsection{Trigger Violations}
So far, the team has not experienced any major violations of the triggers outlined in the charter. All members have remained communicative, met deadlines, and maintained consistent GitHub activity. Minor delays in individual tasks have occurred occasionally, but these were communicated early and resolved collaboratively without impacting the overall progress.

\subsection{Plan to Address Violations}
If future violations occur, the team will follow the three-step escalation process defined in the charter:
\begin{enumerate}
    \item First incident: verbal reminder during team meeting.
    \item Second incident: discussion with the TA to address underlying issues.
    \item Third or repeated incidents: escalation to the course instructor.
\end{enumerate}

If the team finds that the current triggers are too strict or unclear, they will be revised by team consensus. For example, attendance expectations may be adjusted for legitimate scheduling conflicts, or contribution tracking may be clarified to account for non-coding tasks such as documentation or research.

\section{Additional Productivity Metrics}

\wss{If your team has additional metrics of productivity, please feel free to
add them to this report.}

\end{document}
