\documentclass{article}

\usepackage{float}
\restylefloat{table}

\usepackage{booktabs}

\title{Team Contributions: POC\\\progname}

\author{\authname}

\date{}

\input{../Comments}
%% Common Parts

\newcommand{\progname}{Software Engineering} % PUT YOUR PROGRAM NAME HERE
\newcommand{\authname}{Team \#12, Streamliners
\\ Mahad Ahmed
\\ Abyan Jaigirdar
\\ Prerna Prabhu
\\ Farhan Rahman
\\ Ali Zia} % AUTHOR NAMES                  

\usepackage{hyperref}
    \hypersetup{colorlinks=true, linkcolor=blue, citecolor=blue, filecolor=blue,
                urlcolor=blue, unicode=false}
    \urlstyle{same}     

\begin{document}

\maketitle

This document summarizes the contributions of each team member up to the POC
Demo.  The time period of interest is the time between the beginning of the term
and the POC demo.

\section{Demo Plans}

\wss{What will you be demonstrating}

\section{Team Meeting Attendance}

\begin{table}[H]
\centering
\begin{tabular}{ll}
\toprule
\textbf{Student} & \textbf{Meetings}\\
\midrule
Total & 7\\
Mahad Ahmed & 7\\
Abyan Jaigirdar & 7\\
Prerna Prabhu & 7\\
Farhan Rahman & 7\\
Ali Zia & 7\\
\bottomrule
\end{tabular}
\end{table}

\noindent
Our team meets weekly at the start of each deliverable, with additional meetings scheduled as needed, typically in the days leading up to major deadlines. Since the team was formed during Week 3, we have held a total of seven meetings to date, with the exception of reading week. Every team member has been able to attend these meetings, even if occasionally arriving slightly late or needing to leave early.


\section{Supervisor/Stakeholder Meeting Attendance}

\wss{For each team member how many supervisor/stakeholder team meetings have
they attended over the time period of interest.  This number should be determined
from the supervisor meeting issues in the team's repo.  The first entry in the
table should be the total number of supervisor and team meetings held by the
team.  If there is no supervisor, there will usually be meetings with
stakeholders (potential users) that can serve a similar purpose.}

\noindent \textbf{Supervisor's Name: } [fill in this information]

\begin{table}[H]
\centering
\begin{tabular}{ll}
\toprule
\textbf{Student} & \textbf{Meetings}\\
\midrule
Total & Num\\
Name 1 & Num\\
Name 2 & Num\\
Name 3 & Num\\
Name 4 & Num\\
Name 5 & Num\\
\bottomrule
\end{tabular}
\end{table}

\wss{If needed, an explanation for the counts can be provided here.}

\section{Lecture Attendance}

\wss{For each team member how many lectures have they attended over the time
period of interest.  This number should be determined from the lecture issues in
the team's repo. You can find the number of lectures in the time period of
interest by looking at the
\href{https://calendar.google.com/calendar/u/0/embed?src=rnboqiaki1k2la7rpu3bn0um58@group.calendar.google.com&ctz=America/Toronto}
{Google calendar} for the capstone course.}

\wss{NOTE: There will be approximately 13 lectures between the start of class
and the POC demos}

\begin{table}[H]
\centering
\begin{tabular}{ll}
\toprule
\textbf{Student} & \textbf{Lectures}\\
\midrule
Total & Num\\
Name 1 & Num\\
Name 2 & Num\\
Name 3 & Num\\
Name 4 & Num\\
Name 5 & Num\\
\bottomrule
\end{tabular}
\end{table}

\wss{If needed, an explanation for the lecture attendance can be provided here.}

\section{TA Document Discussion Attendance}



\noindent \textbf{TA's Name: } Tiago de Moraes Machado

\begin{table}[H]
\centering
\begin{tabular}{ll}
\toprule
\textbf{Student} & \textbf{Lectures}\\
\midrule
Total & 3\\
Ali Zia & 2\\
Abyan Jaigirdar & 2\\
Mahad Ahmed & 2\\
Farhan Rahman & 2\\
Prerna Prabhu & 2\\
\bottomrule
\end{tabular}
\end{table}

\noindent
There were approximately three meetings held with the TA during this period. Our team attended two of these meetings together. By group consensus, we decided not to attend the most recent session due to a midterm scheduled that evening, as we prioritized the additional study and commute time required on that day.

\section{Commits}

\begin{table}[H]
\centering
\begin{tabular}{lll}
\toprule
\textbf{Student} & \textbf{Commits} & \textbf{Percent}\\
\midrule
Total & 201 & 100\% \\
Mahad Ahmed & 41 & 20.40\% \\
Abyan Jaigirdar & 34 & 16.92\% \\
Prerna Prabhu & 48 & 23.87\% \\
Farhan Rahman & 38 & 18.91\% \\
Ali Zia & 40 & 19.90\% \\
\bottomrule
\end{tabular}
\end{table}

\section{Issue Tracker}

\begin{table}[H]
\centering
\begin{tabular}{ll}
\toprule
\textbf{Student}  & \textbf{Issues Assigned}\\
\midrule
Mahad Ahmed  & 16 \\
Abyan Jaigirdar  & 15 \\
Prerna Prabhu  & 24 \\
Farhan Rahman  & 16 \\
Ali Zia & 16 \\
\bottomrule
\end{tabular}
\end{table}

\noindent
One team member created and opened all issues for upcoming milestones and document sections. This was done intentionally to maintain a consistent issue structure and keep everything organized. During team meetings, these pre-created issues were reviewed based on how much time it would take and then assigned to team members evenly. \\

\noindent
Because issue creation was done in this way, the column "O+C" (Opened and Closed Issues) does not accurately reflect the individual contribution. Instead, the key metric is just the number of issues assigned to each team member. Therefore, we have chosen to remove the "O+C" column and report only the Assigned (Closed) issues.

\section{CICD}

Continuous Integration and Continuous Deployment (CI/CD) will be used to keep the project stable and easy to maintain as new features are added. The team will use GitHub Actions to automatically build, test, and deploy the system.

For \textbf{Continuous Integration (CI)}, every time a pull request is opened, the pipeline will automatically run linting, type checks, and unit tests. This helps catch issues early and ensures that the codebase stays clean and consistent. It will also build the project to make sure that new changes do not break existing functionality before they are merged.

For \textbf{Continuous Deployment (CD)}, after merging into the \texttt{dev} branch, a staging build will be automatically deployed for testing and feedback. When changes are merged into the \texttt{main} branch, a production build will be deployed to the university-hosted environment. This keeps updates smooth and reduces the chance of deployment errors.

GitHub Actions will also send build or test failure notifications to the team so that problems can be fixed quickly. Environment variables and API keys will be stored securely using GitHub’s built-in secret management. Over time, the team may expand the pipeline to include integration or end-to-end tests, but for now the main goal is to automate testing and deployment to save time and improve reliability.

\section{Team Charter Trigger Items}

\wss{Provide a summary of the quantified triggers identified in the team's
charter.}

\wss{Provide a list of any violations of the triggers.  If the team wishes, the
violations can be summarized on aggregate, instead of naming specific team
members.}

\wss{Provide a plan to address the violations.  This could include revising the
triggers, if they are found to be too weak, strong or ambiguous.}

\section{Additional Productivity Metrics}

\wss{If your team has additional metrics of productivity, please feel free to
add them to this report.}

\end{document}