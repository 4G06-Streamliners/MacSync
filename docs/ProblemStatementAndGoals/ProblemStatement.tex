\documentclass{article}

\usepackage{tabularx}
\usepackage{booktabs}
\usepackage{array}

\title{Problem Statement and Goals\\\progname}

\author{\authname}

\date{}

\input{../Comments}
%% Common Parts

\newcommand{\progname}{Software Engineering} % PUT YOUR PROGRAM NAME HERE
\newcommand{\authname}{Team \#12, Streamliners
\\ Mahad Ahmed
\\ Abyan Jaigirdar
\\ Prerna Prabhu
\\ Farhan Rahman
\\ Ali Zia} % AUTHOR NAMES                  

\usepackage{hyperref}
    \hypersetup{colorlinks=true, linkcolor=blue, citecolor=blue, filecolor=blue,
                urlcolor=blue, unicode=false}
    \urlstyle{same}     

\begin{document}
\sloppy

\maketitle

\begin{table}[hp]
\caption{Revision History} \label{TblRevisionHistory}
\begin{tabularx}{\textwidth}{llX}
\toprule
\textbf{Date} & \textbf{Developer(s)} & \textbf{Change}\\
\midrule
2025-09-20 & Prerna Prabhu  & Initial draft of Problem Statement and Goals\\
2025-11-17 & Prerna Prabhu & Addressed TA feedback, updated project deliverables and added security and privacy considerations \\
\bottomrule
\end{tabularx}
\end{table}

\section{Problem Statement}

\subsection{Problem}
The McMaster Engineering Society (MES) organizes large-scale events such as Fireball Formal, Graduation Formal,
 and Pub Nights, each involving several hundred students. Currently, registration, ticketing, waiver collection,
 and check-in processes are fragmented across multiple platforms including Google Forms, spreadsheets, Discord
 announcements, Instagram posts, and manual bookkeeping methods. This fragmentation leads to duplicated work,
 lost information, inconsistent communication, and significant administrative overhead for student organizers,
creaing a confusing, inconsistent experience for attendees.
\\
\\Attendees struggle to:
\begin{itemize}
    \item Access clear, accurate, and consolidated event information.
    \item Register efficiently for events and related logistics (tables, buses, waivers) in a single place.
    \item Receive timely notifications, updates, confirmations, or reminders.
\end{itemize}
Organizers struggle to:
\begin{itemize}
    \item Track ticket sales, waivers, waitlists, registrations, and attendee data across multiple platforms.
    \item Manage accessibility/dietary information effectively.
    \item Minimize repetitive manual effort across multiple tools.
    \item Avoid manual errors and overbooking.
    \item Maintain clear audit trails for finance, risk, and reporting.
\end{itemize}
Ultimately, the absence of a centralized and reliable platform reduces event quality, increases volunteer workload,
and creates stress for both attendees and organizers.

\subsection{Project Deliverables}

This capstone project is part of a larger multi-team initiative to build a full
MES event management platform. Three teams (A, B, and C) are jointly developing
different components of the system. The deliverables listed in this document
reflect the specific scope assigned to our team, and does not represent the
entirety of the platform being built across all groups.\\

\noindent The main objective of our team is to deliver event functionality related to:

\begin{itemize}
    \item registration and ticketing,
    \item secure payment processing,
    \item RBAC/FBAC permissions for organizers,
    \item bus and table capacity-based sign-ups,
    \item analytics relevant to the features we own.
\end{itemize}

\noindent This section explicitly captures the deliverables our team is responsible
for within the broader multi-team project. It clarifies boundaries,
dependencies, and expectations while ensuring alignment with the overall MES
platform architecture.

\subsection{Additional Considerations}

Because this system handles personal and financial information, security and privacy are major concerns.
The system must ensure responsible handling of:
\begin{itemize}
    \item student personal details,
    \item dietary and accessibility information,
    \item financial transactions,
    \item check-in data.
\end{itemize}
\noindent All features must meet MES expectations for confidentiality and reliability.\\

\noindent The project will primarily deliver a web-based platform with a mobile-first attendee experience, though a
standalone mobile app may be explored if time allows. Integration with other university systems (e.g., Outlook
calendar integrations, single sign-on) is not within our team's direct scope, but the design allows for potential
future compatibility.


\subsection{Inputs and Outputs}

\paragraph{Inputs:}
\begin{itemize}
    \item Student event registrations (tickets, RSVPs, bus signups, table preferences).
    \item Waiver acknowledgements and personal details (e.g., dietary and accessibility requirements).
    \item Payment details for ticket purchases.
    \item Admin inputs for event setup (ticket types, capacities, schedules, notifications).
\end{itemize}


\paragraph{Outputs:}
\begin{itemize}
    \item Confirmation of event registration and digital tickets/QR codes.
    \item Notifications and reminders about events.
    \item Waitlist updates and allocations.
    \item Admin dashboards showing ticket sales, attendee demographics, accessibility/dietary data, and financial tracking.
    \item Check-in validation at event entry points.
\end{itemize}


\subsection{Stakeholders}
\paragraph{Primary Stakeholders:}
\begin{itemize}
    \item MES Event Attendees (students): Use the platform to register, purchase tickets, receive updates,
        and check in.
    \item MES Event Organizers/Volunteers: Use the platform to manage events, track registrations, ticket
     sales and payments, waivers, handle check-ins, and communicate with attendees.
\end{itemize}

\paragraph{Secondary Stakeholders:}
\begin{itemize}
    \item MES Executives: Oversee finances, risk management, and reporting.
\end{itemize}

\paragraph{Tertiary Stakeholders:}
\begin{itemize}
    \item Sponsors and partners: Interested in gaining visibility, engagement, and smooth execution.
    \item McMaster University Administration: Indirectly involved for compliance, liability via waivers,
     and student satisfaction.
\end{itemize}

\subsection{Environment}
\paragraph{Hardware Environment:}
\begin{itemize}
    \item Attendees: Smartphones (iOS/Android) and laptops for registration, notifications, and check-in.
    \item Organizers: Laptops/desktops for backend dashboards, mobile devices for on-site management and QR code scanning.
\end{itemize}

\paragraph{Software Environment:}
\begin{itemize}
    \item Web-based admin dashboard for event creation and analytics.
    \item Cross-platform mobile app for students (primary focus) with fallback web access.
    \item Payment integration (e.g., Stripe, Square, Paypal).
    \item Backend database to store user, event, and financial data.
\end{itemize}


\section{Goals}


\paragraph{Centralized Registration \& Ticketing}
\begin{itemize}
    \item To reduce administrative overhead for MES organizers by 40\% (measured through reported time spent on registration tasks) by consolidating ticketing, RSVPs, and event details into one platform, in order to eliminate confusion caused by fragmented tools.
\end{itemize}

\paragraph{Secure Payment Integration}
\begin{itemize}
    \item To enable reliable, PCI-compliant payments for MES events with a transaction failure rate below 5\% during peak sales, in order to provide safe, trusted, and audit-friendly financial workflows.
\end{itemize}


\paragraph{Role-Based Access Control (RBAC/FBAC)}
\begin{itemize}
    \item To reduce organizer errors caused by misconfigured access by at least 60\% (tracked through internal issue reports) by implementing RBAC/FBAC, in order to ensure secure and efficient administrative workflows.
\end{itemize}

\paragraph{Bus \& Table Sign-ups}
\begin{itemize}
\item To eliminate overbooking incidents (target: 0 occurrences) and reduce logistics coordination time by 50\% by providing automated holds, allocations, and real-time capacity validation.
\end{itemize}

\paragraph{Notifications \& Reminders}
\begin{itemize}
    \item To decrease attendee no-shows by 20\% (measured across comparable MES events) by sending automated notifications, confirmations, and reminders.
\end{itemize}

\paragraph{Analytics \& Reporting Tools}
\begin{itemize}
    \item To reduce MES executive reporting time by 70\% by providing dashboards for finances, demographics, accessibility data, and ticket metrics, enabling more informed planning.
\end{itemize}

\paragraph{Attendee User Experience}
\begin{itemize}
    \item To achieve an 85\% or higher satisfaction score in pilot testing by delivering a mobile-first interface that is accessible, intuitive, and easy to navigate.
\end{itemize}

\paragraph{Security and Privacy}
\begin{itemize}
    \item To protect student personal, accessibility, and financial information with zero security/privacy incidents during the pilot period, in order to maintain trust and meet MES and university expectations for confidentiality and responsible data handling.
\end{itemize}

\section{Stretch Goals}

\begin{itemize}
    \item Personalized event recommendations based on student interests.
    \item Calendar integration (Google/Outlook/Apple).
    \item Post-event engagement features (photo gallery, feedback surveys, lost \& found updates).
    \item Sponsor visibility tools (logos, surveys, branded features).
    \item Dietary/accessibility matching algorithms for meal and seating planning.
\end{itemize}

\section{Extras}

\begin{itemize}
    \item Usability report and testing with MES organizers and student participants.
    \item User documentation for both attendees and admin organizers.
\end{itemize}

\newpage

\section*{Appendix --- Reflection}


\input{../Reflection.tex}

\begin{enumerate}
    \item What went well while writing this deliverable?
    \\
    \\The team was able to understand the problem quickly because of each member's familiarity with MES events and how they are currently run.
     From our own experiences, we already understood the main pain points as well as the features that would be most valuable
     to both organizers and attendees. This made it easier to distinguish between core goals of the project and additional features. Since we
     had a clear outline of what the system should do provided by the supervisor, our team was able to divide tasks quickly and effectively
      for this deliverable. Collaboration went smoothly, and we tackled the work for this deliverable in a fast and organized way. Having shared context
      gave us confidence in identifying the users' needs and building a well-structured document. Being able to use github issues to organize and divide
      the work also helped us stay on track and ensure that everyone contributed equally.
    \item What pain points did you experience during this deliverable, and how
    did you resolve them?
    \\
    \\One of the biggest challenges was that we did not know which project we would be assigned until very close to the deadline.
     Especially, since this project is not a traditional capstone project, in the way that it is divided amongst two other groups, and the scope is quite large.
     This made it difficult to start early and limited the amount of feedback we could receive from the TA before submission. Even
      though we already understood the problem well, it was sometimes difficult to frame our knowledge at a high-level overview
     instead of diving into details. Another challenge was technical: setting up LaTeX and repositories was new for some of us and
      was time-consuming to get familiar with. We also struggled to clearly differentiate between goals vs. stretch goals, and 
      primary vs. secondary stakeholders (especially since both students and organizers are highly active users). Finally, balancing
     the level of detail such as being specific enough to show thought in requirements, but high-level enough to avoid over-scoping was an ongoing
      challenge. We resolved these issues through group discussions, reviewing example documents, and iterating on the structure.

    \item How did you and your team adjust the scope of your goals to ensure
    they are suitable for a Capstone project (not overly ambitious but also of
    appropriate complexity for a senior design project)?
    \\
    \\At the beginning, we had many ambitious ideas, including projects involving AI and large language models, as well as other advanced
     projects that were highly related to specific team members' interests. However, we realized
     that not all team members had the technical background or confidence to pursue those ideas effectively. This made us step back
     and focus on a project that was both practical and meaningful to all of us. The MES event management platform came naturally from
     our own past experiences and frustrations, which made it more grounded and relatable. We also discussed how it could connect to
     our co-op experiences: it would challenge us to learn new skills such as payment integration, role-based access control, and
     mobile app development, which some of us have prior experience in while others will be completely new to these concepts,
     while still refining technical skills we already had from previous work terms. By focusing on a project
     with clear real-world impact but narrowing it to a feasible scope, we ensured it was the right balance of ambitious and achievable
     for a capstone. Additionally, since the project is divided among two other groups, we made sure to clearly define our team's specific
      responsibilities and the features that we can work on without overlapping or depending on other groups' contributions. This way we are
      able to manage our workload effectively and ensure that we can deliver a complete and functional product within the given timeframe and
      not be blocked by other teams' progress or takuing on too many features.
\end{enumerate}  

\end{document}