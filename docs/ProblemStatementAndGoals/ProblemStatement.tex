\documentclass{article}

\usepackage{tabularx}
\usepackage{booktabs}
\usepackage{array}

\title{Problem Statement and Goals\\\progname}

\author{\authname}

\date{}

%% Comments

\usepackage{color}

\newif\ifcomments\commentstrue %displays comments
%\newif\ifcomments\commentsfalse %so that comments do not display

\ifcomments
\newcommand{\authornote}[3]{\textcolor{#1}{[#3 ---#2]}}
\newcommand{\todo}[1]{\textcolor{red}{[TODO: #1]}}
\else
\newcommand{\authornote}[3]{}
\newcommand{\todo}[1]{}
\fi

\newcommand{\wss}[1]{\authornote{magenta}{SS}{#1}} 
\newcommand{\plt}[1]{\authornote{cyan}{TPLT}{#1}} %For explanation of the template
\newcommand{\an}[1]{\authornote{cyan}{Author}{#1}}

%% Common Parts

\newcommand{\progname}{ProgName} % PUT YOUR PROGRAM NAME HERE
\newcommand{\authname}{Team \#, Team Name
\\ Student 1 name
\\ Student 2 name
\\ Student 3 name
\\ Student 4 name} % AUTHOR NAMES                  

\usepackage{hyperref}
    \hypersetup{colorlinks=true, linkcolor=blue, citecolor=blue, filecolor=blue,
                urlcolor=blue, unicode=false}
    \urlstyle{same}
                                


\begin{document}
\sloppy

\maketitle

\begin{table}[hp]
\caption{Revision History} \label{TblRevisionHistory}
\begin{tabularx}{\textwidth}{llX}
\toprule
\textbf{Date} & \textbf{Developer(s)} & \textbf{Change}\\
\midrule
Date1 & Name(s) & Description of changes\\
Date2 & Name(s) & Description of changes\\
... & ... & ...\\
\bottomrule
\end{tabularx}
\end{table}

\section{Problem Statement}

\subsection{Problem}
The McMaster Engineering Society (MES) organizes large-scale events such as Fireball Formal, Graduation Formal,
 and Pub Nights, often drawing several hundred attendees. Currently, registration, ticketing, waiver collection,
 and check-in processes are fragmented across multiple platforms (Google Forms, spreadsheets, Discord, Instagram,
 etc.). This lack of centralization increases the administrative burden on student organizers and creates a
 confusing, inconsistent experience for attendees.
\\
\\Students struggle to:
\begin{itemize}
    \item Access clear, consolidated event information.
    \item Register efficiently for events and related logistics (tables, buses).
    \item Receive timely updates or reminders.
\end{itemize}
Organizers struggle to:
\begin{itemize}
    \item Manage ticket sales, waitlists, and attendee data in one place.
    \item Track accessibility/dietary requirements effectively.
    \item Minimize repetitive manual effort across multiple tools.
\end{itemize}
The absence of a centralized system reduces event memorability, increases missed opportunities, and 
wastes volunteer time that could be spent on event quality rather than administration.

\subsection{Inputs and Outputs}

\paragraph{Inputs:}
\begin{itemize}
    \item Student event registrations (tickets, RSVPs, bus signups, table preferences).
    \item Waiver acknowledgements and personal details (e.g., dietary and accessibility requirements).
    \item Payment details for ticket purchases.
    \item Admin inputs for event setup (ticket types, capacities, schedules, notifications).
\end{itemize}


\paragraph{Outputs:}
\begin{itemize}
    \item Confirmation of event registration and digital tickets/QR codes.
    \item Notifications and reminders about events.
    \item Waitlist updates and allocations.
    \item Admin dashboards showing ticket sales, attendee demographics, accessibility/dietary data, and financial tracking.
    \item Check-in validation at event entry points.
\end{itemize}


\subsection{Stakeholders}
\paragraph{Primary Stakeholders:}
\begin{itemize}
    \item MES Event Attendees (students): Use the platform to register, purchase tickets, receive updates,
        and check in.
    \item MES Event Organizers/Volunteers: Use the platform to manage events, track registrations, ticket
     sales and payments, waivers, handle check-ins, and communicate with attendees.
\end{itemize}

\paragraph{Secondary Stakeholders:}
\begin{itemize}
    \item MES Executives: Oversee finances, risk management, and reporting.
\end{itemize}

\paragraph{Tertiary Stakeholders:}
\begin{itemize}
    \item Sponsors and partners: Interested in gaining visibility, engagement, and smooth execution.
    \item McMaster University Administration: Indirectly involved for compliance, liability via waivers,
     and student satisfaction.
\end{itemize}

\subsection{Environment}
\paragraph{Hardware Environment:}
\begin{itemize}
    \item Attendees: Smartphones (iOS/Android) and laptops for registration, notifications, and check-in.
    \item Organizers: Laptops/desktops for backend dashboards, mobile devices for on-site management and QR code scanning.
\end{itemize}

\paragraph{Software Environment:}
\begin{itemize}
    \item Web-based admin dashboard for event creation and analytics.
    \item Cross-platform mobile app for students (primary focus) with fallback web access.
    \item Payment integration (e.g., Stripe, Square, Paypal).
    \item Backend database to store user, event, and financial data.
\end{itemize}


\section{Goals}


\paragraph{Centralized Registration \& Ticketing}
\begin{itemize}
    \item \textbf{Description:} The platform must consolidate ticket purchasing, registration, and RSVPs into one place.
    \item \textbf{Justification:} This eliminates confusion caused by scattered tools and ensures students always have access to the latest event information.
\end{itemize}

\paragraph{Payment Integration}
\begin{itemize}
    \item \textbf{Description:} Provide secure and flexible payment options (Stripe, Square, PayPal).
    \item \textbf{Justification:} Enables students to pay quickly with widely used systems while reducing cash-handling risks for organizers.
\end{itemize}

\paragraph{Role-Based Access Control (RBAC/FBAC)}
\begin{itemize}
    \item \textbf{Description:} Implement granular permissions so organizers only see/manage the tools relevant to their role.
    \item \textbf{Justification:} Reduces errors, ensures security, and improves efficiency in large event teams.
\end{itemize}

\paragraph{Bus \& Table Sign-ups}
\begin{itemize}
    \item \textbf{Description:} Allow attendees to reserve buses and tables with automatic capacity tracking.
    \item \textbf{Justification:} Simplifies logistics and avoids overbooking, replacing spreadsheets and manual coordination.
\end{itemize}

\paragraph{Notifications \& Reminders}
\begin{itemize}
    \item \textbf{Description:} Send push notifications and reminders for registrations, updates, and cancellations.
    \item \textbf{Justification:} Improves event memorability, prevents missed opportunities, and reduces no-shows.
\end{itemize}

\paragraph{Analytics \& Reporting}
\begin{itemize}
    \item \textbf{Description:} Provide organizers with real-time dashboards on sales, demographics, and waitlists.
    \item \textbf{Justification:} Supports decision-making, improves resource allocation, and helps executives evaluate event success.
\end{itemize}

\paragraph{Attendee Experience}
\begin{itemize}
    \item \textbf{Description:} Ensure the platform is mobile-first, intuitive, and accessible.
    \item \textbf{Justification:} Reduces barriers to use, supports inclusivity, and maximizes student engagement.
\end{itemize}

\section{Stretch Goals}

\begin{itemize}
    \item Personalized event recommendations based on student interests.
    \item Calendar integration (Google/Outlook/Apple).
    \item Post-event engagement features (photo gallery, feedback surveys, lost \& found updates).
    \item Sponsor visibility tools (logos, surveys, branded features).
    \item Dietary/accessibility matching algorithms for meal and seating planning.
\end{itemize}

\section{Extras}

\wss{For CAS 741: State whether the project is a research project. This
designation, with the approval (or request) of the instructor, can be modified
over the course of the term.}

\wss{For SE Capstone: List your extras.  Potential extras include usability
testing, code walkthroughs, user documentation, formal proof, GenderMag
personas, Design Thinking, etc.  (The full list is on the course outline and in
Lecture 02.) Normally the number of extras will be two.  Approval of the extras
will be part of the discussion with the instructor for approving the project.
The extras, with the approval (or request) of the instructor, can be modified
over the course of the term.}

\newpage

\section*{Appendix --- Reflection}

\wss{Not required for CAS 741}

The purpose of reflection questions is to give you a chance to assess your own
learning and that of your group as a whole, and to find ways to improve in the
future. Reflection is an important part of the learning process.  Reflection is
also an essential component of a successful software development process.  

Reflections are most interesting and useful when they're honest, even if the
stories they tell are imperfect. You will be marked based on your depth of
thought and analysis, and not based on the content of the reflections
themselves. Thus, for full marks we encourage you to answer openly and honestly
and to avoid simply writing ``what you think the evaluator wants to hear.''

Please answer the following questions.  Some questions can be answered on the
team level, but where appropriate, each team member should write their own
response:


\begin{enumerate}
    \item What went well while writing this deliverable?
    \\
    \\The team was able to understand the problem quickly because of each member's familiarity with MES events and how they are currently run.
     From our own experiences, we already understood the main pain points as well as the features that would be most valuable
     to both organizers and attendees. This made it easier to distinguish between core goals of the project and additional features. Since we
     had a clear outline of what the system should do provided by the supervisor, our team was able to divide tasks quickly and effectively
      for this deliverable. Collaboration went smoothly, and we tackled the work for this deliverable in a fast and organized way. Having shared context
      gave us confidence in identifying the users' needs and building a well-structured document. Being able to use github issues to organize and divide
      the work also helped us stay on track and ensure that everyone contributed equally.
    \item What pain points did you experience during this deliverable, and how
    did you resolve them?
    \\
    \\One of the biggest challenges was that we did not know which project we would be assigned until very close to the deadline.
     Especially, since this project is not a traditional capstone project, in the way that it is divided amongst two other groups, and the scope is quite large.
     This made it difficult to start early and limited the amount of feedback we could receive from the TA before submission. Even
      though we already understood the problem well, it was sometimes difficult to frame our knowledge at a high-level overview
     instead of diving into details. Another challenge was technical: setting up LaTeX and repositories was new for some of us and
      was time-consuming to get familiar with. We also struggled to clearly differentiate between goals vs. stretch goals, and 
      primary vs. secondary stakeholders (especially since both students and organizers are highly active users). Finally, balancing
     the level of detail such as being specific enough to show thought in requirements, but high-level enough to avoid over-scoping was an ongoing
      challenge. We resolved these issues through group discussions, reviewing example documents, and iterating on the structure.

    \item How did you and your team adjust the scope of your goals to ensure
    they are suitable for a Capstone project (not overly ambitious but also of
    appropriate complexity for a senior design project)?
    \\
    \\At the beginning, we had many ambitious ideas, including projects involving AI and large language models, as well as other advanced
     projects that were highly related to specific team members' interests. However, we realized
     that not all team members had the technical background or confidence to pursue those ideas effectively. This made us step back
     and focus on a project that was both practical and meaningful to all of us. The MES event management platform came naturally from
     our own past experiences and frustrations, which made it more grounded and relatable. We also discussed how it could connect to
     our co-op experiences: it would challenge us to learn new skills such as payment integration, role-based access control, and
     mobile app development, which some of us have prior experience in while others will be completely new to these concepts,
     while still refining technical skills we already had from previous work terms. By focusing on a project
     with clear real-world impact but narrowing it to a feasible scope, we ensured it was the right balance of ambitious and achievable
     for a capstone. Additionally, since the project is divided among two other groups, we made sure to clearly define our team's specific
      responsibilities and the features that we can work on without overlapping or depending on other groups' contributions. This way we are
      able to manage our workload effectively and ensure that we can deliver a complete and functional product within the given timeframe and
      not be blocked by other teams' progress or takuing on too many features.
\end{enumerate}  

\end{document}