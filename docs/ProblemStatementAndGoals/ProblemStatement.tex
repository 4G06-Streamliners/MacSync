\documentclass{article}

\usepackage{tabularx}
\usepackage{booktabs}

\title{Problem Statement and Goals\\\progname}

\author{\authname}

\date{}

\input{../Comments}
%% Common Parts

\newcommand{\progname}{Software Engineering} % PUT YOUR PROGRAM NAME HERE
\newcommand{\authname}{Team \#12, Streamliners
\\ Mahad Ahmed
\\ Abyan Jaigirdar
\\ Prerna Prabhu
\\ Farhan Rahman
\\ Ali Zia} % AUTHOR NAMES                  

\usepackage{hyperref}
    \hypersetup{colorlinks=true, linkcolor=blue, citecolor=blue, filecolor=blue,
                urlcolor=blue, unicode=false}
    \urlstyle{same}     

\begin{document}

\maketitle

\begin{table}[hp]
\caption{Revision History} \label{TblRevisionHistory}
\begin{tabularx}{\textwidth}{llX}
\toprule
\textbf{Date} & \textbf{Developer(s)} & \textbf{Change}\\
\midrule
Date1 & Name(s) & Description of changes\\
Date2 & Name(s) & Description of changes\\
... & ... & ...\\
\bottomrule
\end{tabularx}
\end{table}

\section{Problem Statement}

\wss{You should check your problem statement with the
\href{https://github.com/smiths/capTemplate/blob/main/docs/Checklists/ProbState-Checklist.pdf}
{problem statement checklist}.} 

\wss{You can change the section headings, as long as you include the required
information.}

\subsection{Problem}

\subsection{Inputs and Outputs}

\wss{Characterize the problem in terms of ``high level'' inputs and outputs.  
Use abstraction so that you can avoid details.}

\subsection{Stakeholders}

\subsection{Environment}

\wss{Hardware and Software Environment}

\section{Goals}

\section{Stretch Goals}

\section{Extras}

\wss{For CAS 741: State whether the project is a research project. This
designation, with the approval (or request) of the instructor, can be modified
over the course of the term.}

\wss{For SE Capstone: List your extras.  Potential extras include usability
testing, code walkthroughs, user documentation, formal proof, GenderMag
personas, Design Thinking, etc.  (The full list is on the course outline and in
Lecture 02.) Normally the number of extras will be two.  Approval of the extras
will be part of the discussion with the instructor for approving the project.
The extras, with the approval (or request) of the instructor, can be modified
over the course of the term.}

\newpage{}

\section*{Appendix --- Reflection}

\wss{Not required for CAS 741}

\input{../Reflection.tex}

\begin{enumerate}
    \item What went well while writing this deliverable?
    \\
    \\We were able to understand the problem quickly because of our familiarity with MES events and how they are currently run.
     From our own experiences, we already understood the main pain points as well as the features that would be most valuable
     to both organizers and attendees. This made it easier to distinguish between core goals and additional features. Since we
      had a clear outline of what the system should do, our team was able to divide tasks quickly and effectively. 
      Collaboration went smoothly, and we tackled the work for this deliverable in a fast and organized way. Having shared context
       gave us confidence in identifying the users’ needs and building a well-structured document.
    \item What pain points did you experience during this deliverable, and how
    did you resolve them?
    \\
    \\One of the biggest challenges was that we did not know which project we would be assigned until very close to the deadline.
     This made it difficult to start early and limited the amount of feedback we could receive from the TA before submission. Even
      though we already understood the problem well, it was sometimes difficult to frame our knowledge at a high-level overview
     instead of diving into details. Another challenge was technical: setting up LaTeX and repositories was new for some of us and
      was time-consuming to get familiar with. We also struggled to clearly differentiate between goals vs. stretch goals, and 
      primary vs. secondary stakeholders (especially since both students and organizers are highly active users). Finally, balancing
     the level of detail such as being specific enough to show detail, but high-level enough to avoid over-scoping was an ongoing
      challenge. We resolved these issues through group discussions, reviewing example documents, and iterating on the structure.

    \item How did you and your team adjust the scope of your goals to ensure
    they are suitable for a Capstone project (not overly ambitious but also of
    appropriate complexity for a senior design project)?
    \\
    \\At the beginning, we had many ambitious ideas, including projects involving AI and large language models. However, we realized
     that not all team members had the technical background or confidence to pursue those ideas effectively. This made us step back
     and focus on a project that was both practical and meaningful to all of us. The MES event management platform came naturally from
     our own past experiences and frustrations, which made it more grounded and relatable. We also discussed how it could connect to
     our co-op experiences: it would challenge us to learn new skills (such as payment integration, role-based access control, and
     mobile app development) while still refining technical skills we already had from previous work terms. By focusing on a project
     with clear real-world impact but narrowing it to a feasible scope, we ensured it was the right balance of ambitious and achievable
     for a capstone.
\end{enumerate}  

\end{document}