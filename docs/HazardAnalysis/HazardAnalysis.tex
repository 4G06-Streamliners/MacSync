\documentclass{article}

\usepackage{booktabs}
\usepackage{tabularx}
\usepackage{hyperref}

\hypersetup{
    colorlinks=true,       % false: boxed links; true: colored links
    linkcolor=red,          % color of internal links (change box color with linkbordercolor)
    citecolor=green,        % color of links to bibliography
    filecolor=magenta,      % color of file links
    urlcolor=cyan           % color of external links
}

\title{Hazard Analysis\\\progname}

\author{\authname}

\date{}

\input{../Comments}
%% Common Parts

\newcommand{\progname}{Software Engineering} % PUT YOUR PROGRAM NAME HERE
\newcommand{\authname}{Team \#12, Streamliners
\\ Mahad Ahmed
\\ Abyan Jaigirdar
\\ Prerna Prabhu
\\ Farhan Rahman
\\ Ali Zia} % AUTHOR NAMES                  

\usepackage{hyperref}
    \hypersetup{colorlinks=true, linkcolor=blue, citecolor=blue, filecolor=blue,
                urlcolor=blue, unicode=false}
    \urlstyle{same}     

\begin{document}

\maketitle
\thispagestyle{empty}

~\newpage

\pagenumbering{roman}

\begin{table}[hp]
\caption{Revision History} \label{TblRevisionHistory}
\begin{tabularx}{\textwidth}{llX}
\toprule
\textbf{Date} & \textbf{Developer(s)} & \textbf{Change}\\
\midrule
2025-10-02 & Abyan & Scope and Purpose\\
2025-10-02 & Farhan & Critical assumptions made for the project\\
2025-10-01 & Prerna & Initial draft of safety and security requirements\\
... & ... & ...\\
\bottomrule
\end{tabularx}
\end{table}

~\newpage

\tableofcontents

~\newpage

\pagenumbering{arabic}

\wss{You are free to modify this template.}

\section{Introduction}

\wss{You can include your definition of what a hazard is here.}

\section{Scope and Purpose of Hazard Analysis}


The purpose of this hazard analysis is to identify and evaluate potential hazards that could arise during 
the development and operation of the Large Event Management System (LEMS). Since LEMS is a 
software-based platform that supports event registration, ticketing, payments, and participant 
management, hazards are primarily related to data integrity, system availability, security, and user 
interactions. By analyzing these risks early, the project team can define mitigation strategies, 
incorporate safety and security requirements into the design, and reduce the likelihood of 
organizational or reputational harm.

\par
\vspace{1em}

The scope of this analysis covers all major components of LEMS, including the backend services, 
web and mobile applications, and the database. It considers hazards introduced by user error, software 
defects, integration failures across modules, and security vulnerabilities. Hazards that fall outside the 
team’s control, such as third-party cloud hosting failures or issues with external payment providers 
(e.g., Stripe), are acknowledged but not analyzed in detail.

\par
\vspace{1em}

The goals of this hazard analysis are:
\begin{itemize}
  \item To proactively identify risks related to security, data integrity, privacy, and availability in LEMS.
  \item To evaluate the potential consequences of failures, such as data loss, financial errors, or 
        unauthorized access to sensitive information.
  \item To define mitigations or controls that reduce these risks to acceptable levels.
  \item To guide architectural and design decisions and support the definition of non-functional 
        requirements such as reliability, security, and maintainability.
\end{itemize}

This hazard analysis will be refined as the project design evolves, ensuring that new risks are 
addressed as they emerge. It is part of the broader quality assurance process and ensures that LEMS 
meets the standards of safety, security, and reliability expected by the McMaster Engineering Society.

\section{System Boundaries and Components}

\wss{Dividing the system into components will help you brainstorm the hazards.
You shouldn't do a full design of the components, just get a feel for the major
ones.  For projects that involve hardware, the components will typically include
each individual piece of hardware.  If your software will have a database, or an
important library, these are also potential components.}

\section{Critical Assumptions}

The following assumptions have been identified as critical to the safe and reliable operation of the MacSync platform.

\begin{itemize}
    \item \textbf{Reliable Internet Access:} It is assumed that both attendees and organizers will have access to stable internet connections during registration, payment, and check-in. While temporary connectivity loss may occur, the system must handle these cases gracefully.
    
    \item \textbf{Third-Party Service Availability:} The platform depends on external services such as payment processors (e.g., Stripe, PayPal) and hosting infrastructure. It is assumed these services provide high availability, but the system will still account for outages or delays to prevent complete operational failure.
    
    \item \textbf{Device Compatibility:} It is assumed that attendees will primarily use modern smartphones and organizers will have access to laptops or mobile devices capable of running the dashboard. Reliance on outdated devices or unsupported browsers must be minimized through compatibility testing.
    
    \item \textbf{User Data Accuracy:} The system assumes that users provide correct information (e.g., dietary restrictions, accessibility needs, payment details). However, hazards tied to incorrect or incomplete inputs will be addressed by validation checks.
    
    \item \textbf{Organizational Oversight:} It is assumed that event organizers will actively monitor the system for anomalies (e.g., payment disputes, capacity errors, failed notifications). This system will assist and automate many of the tasks and centralize information, but human oversight is still necessary to manage unexpected situations.
    
    \item \textbf{Security Measures:} It is assumed that standard security practices (encrypted storage, secure authentication, and role-based access control) will be implemented and maintained. Failure to enforce these could expose sensitive student data or enable fraudulent event access.
\end{itemize}


\section{Failure Mode and Effect Analysis}

\wss{Include your FMEA table here. This is the most important part of this document.}
\wss{The safety requirements in the table do not have to have the prefix SR.
The most important thing is to show traceability to your SRS. You might trace to
requirements you have already written, or you might need to add new
requirements.}
\wss{If no safety requirement can be devised, other mitigation strategies can be
entered in the table, including strategies involving providing additional
documentation, and/or test cases.}

\section{Safety and Security Requirements}

\subsection{Data Security Requirements}
\begin{itemize}
    \item \textbf{SSR-DS1 (Encryption in Transit):} The system shall use TLS 1.2 or higher to encrypt all network communications between client applications, servers, and third-party APIs.
    \item \textbf{SSR-DS2 (Encryption at Rest):} The system shall encrypt sensitive stored data (user profiles, accessibility needs, dietary restrictions, waiver records) using AES-256.
    \item \textbf{SSR-DS3 (Privacy Compliance):} The system shall comply with PIPEDA and relevant McMaster University policies regarding data collection, storage, and retention.
    \item \textbf{SSR-DS4 (Data Minimization):} The system shall only store the minimum personal data necessary for event operations (e.g., no raw payment card details).
\end{itemize}

\subsection{Payment Security Requirements}
\begin{itemize}
    \item \textbf{SSR-PS1 (Third-Party Gateways):} All financial transactions shall be processed exclusively through secure payment providers (Stripe, Square, or PayPal). The system shall not store or transmit raw credit card details.
    \item \textbf{SSR-PS2 (Transaction Confirmation):} The system shall generate a unique transaction confirmation code for every payment, retrievable by both the attendee and the organizer.
    \item \textbf{SSR-PS3 (Audit Logging):} All financial operations (payments, refunds, chargebacks) shall be logged in an immutable audit trail accessible only to authorized financial officers.
    \item \textbf{SSR-PS4 (Duplicate Protection):} The system shall prevent duplicate payments by checking for transaction IDs before ticket issuance.
\end{itemize}

\subsection{Access Control Requirements}
\begin{itemize}
    \item \textbf{SSR-AC1 (Role-Based Access Control):} The system shall enforce RBAC/FBAC to ensure that each organizer only has access to the features necessary for their role.
    \item \textbf{SSR-AC2 (Authentication):} The system shall integrate with McMaster University's Single Sign-On (SSO) to authenticate student attendees and organizers.
    \item \textbf{SSR-AC3 (Session Management):} The system shall automatically expire inactive sessions after 15 minutes of inactivity for admin accounts and 60 minutes for attendee accounts.
    \item \textbf{SSR-AC4 (Privilege Escalation Protection):} The system shall log and alert MES executives of any attempts at unauthorized access or privilege escalation.
\end{itemize}

\subsection{System Reliability \& Availability Requirements}
\begin{itemize}
    \item \textbf{SSR-SR1 (Uptime):} The system shall maintain at least 98\% uptime during active registration and event check-in periods.
    \item \textbf{SSR-SR2 (Offline Check-in):} The system shall support offline QR code validation to allow event entry if internet connectivity is unavailable.
    \item \textbf{SSR-SR3 (Notification Redundancy):} All time-sensitive notifications (registration confirmations, event reminders) shall include retry mechanisms to ensure delivery.
    \item \textbf{SSR-SR4 (Backup \& Recovery):} The system shall automatically back up all event and registration data daily and enable recovery within 24 hours of data loss.
\end{itemize}

\subsection{Operational Safety Requirements}
\begin{itemize}
    \item \textbf{SSR-OS1 (Waiver Enforcement):} The system shall not allow final ticket confirmation until the attendee has digitally signed the required waiver.
    \item \textbf{SSR-OS2 (Capacity Validation):} The system shall enforce real-time capacity limits for bus sign-ups, table sign-ups, and RSVP sign-ups to prevent overbooking.
    \item \textbf{SSR-OS3 (Organizer Accountability):} All organizer actions (e.g., modifying capacities, issuing refunds, editing registrations) shall be logged with a timestamp and user identifier.
    \item \textbf{SSR-OS4 (Fail-Safe Defaults):} In the event of a system error, the system shall default to denying access to sensitive data until the error is resolved.
\end{itemize}


\section{Roadmap}

\wss{Which safety requirements will be implemented as part of the capstone timeline?
Which requirements will be implemented in the future?}

\newpage{}

\section*{Appendix --- Reflection}

\wss{Not required for CAS 741}

\input{../Reflection.tex}

\begin{enumerate}
    \item What went well while writing this deliverable? 
    \item What pain points did you experience during this deliverable, and how
    did you resolve them?
    \item Which of your listed risks had your team thought of before this
    deliverable, and which did you think of while doing this deliverable? For
    the latter ones (ones you thought of while doing the Hazard Analysis), how
    did they come about?
    \item Other than the risk of physical harm (some projects may not have any
    appreciable risks of this form), list at least 2 other types of risk in
    software products. Why are they important to consider?
\end{enumerate}

\end{document}